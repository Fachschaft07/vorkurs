\subsection{Warum Mathematik im Informatikstudium?}
Vielen Studierenden ist nicht klar, warum Mathematik so grundlegend wichtig für ihr Studium ist - schließlich studieren sie ja keine (reine) Mathematik! Dieses Kapitel soll jenes Verständnisproblem lösen und aufzeigen, welche Schnittpunkte zwischen Mathematik und Informatik bestehen.\\
Mit Mathematik haben Informatiker ein zusätzliches, sehr mächtiges Werkzeug zur Hand, mit dem viele scheinbar unlösbare Probleme plötzlich lösbar werden.
Deshalb gibt es in der Informatik einige Bereiche in denen ein gewisses mathematisches Grundverständnis notwendig ist.\\
Im Folgenden möchten wir nun anhand einiger Beispiele den Nutzen von Mathematik-Kenntnissen im Bereich Informatik verdeutlicht:

\subsubsection{Spieleentwicklung}
Die Entwicklung eines Spiels umfasst neben der Programmierung noch sehr viele andere Bereiche, wie Psychologie, Design und Soziologie (um nur ein paar zu nennen). Auch die Mathematik spielt hierbei eine große Rolle:\\
So ermöglicht es einem die (nummerische) Integration, dass sich Figuren realistischer Bewegen können.\\
Auch Lichteffekte wie Spiegelungen oder der Soundeffekte wie Widerhall müssen zuerst korrekt mathematisch beschrieben und dann implementiert werden, wenn sie \glqq realistisch\grqq wirken sollen.\\
Alle beschriebenen Probleme können mit relativ einfachen, mathematischen Methoden gelöst werden.

\subsubsection{Business Analytics}
Business Analytics wird eingesetzt, um die Probleme in der Unternehmenswelt mittels Analyse großer Datenmengen (Big Data) zu lösen.\\
Solche Probleme sind zum Beispiel die Ermittlung von \glqq Risikokunden\grqq, die unzufrieden sind und deshalb ihren Handyvertrag kündigen wollen. 
Können diese Kunden per Business Analytics gezielt ermittelt werden, so ist es möglich sie mit speziellen Angeboten, wie Freiminuten, von der Kündigung abzuhalten.\\
Die Ermittlung der Kunden erfolgt hierbei durch statistische Analysen wie Regression oder Decision Trees.

\subsubsection{Machine Learning}
Durch die immer schneller rechnenden Computer können heute Probleme gelöst werden, die früher nur Menschen erledigen konnten. Dazu gehören automatische Gesichtserkennung, das Auffinden von Tumoren in medizinischen Bildern oder die Erkennung von Kreditkartenbetrug. Meist steht hinter diesen Anwendungen das sogenannte \glqq Machine Learning \grqq, was bedeutet, dass die Maschine (der Computer) selbst lernt, welche Eigenschaften in den auftretenden Daten (Bilder von Gesichtern, Kreditkartenüberweisungen, Bilder von Tumoren oder gesundem Gewebe) zu welchem Ergebnis passt (Gesichter werden bestimmten Menschen zugeordnet, eine überweisung wird als Betrug markiert, ein Gewebebild wird als gesund gekennzeichnet). Das praktische am selbstlernenden Computer: der Mensch muss nur noch eine sogenannte Trainingsmenge an Datenmaterial bereitstellen, an der der Computer lernen kann - nach dem Training ist die Maschine selbständig in der Lage, eine Zuordnung zu machen.
Diese selbstlernenden Programme basieren auf mathematische Grundlagen.\\
Einen hierfür oft verwendeten Ansatz bieten die sogenannten \glqq neuronalen Netze\grqq.
Welche wiederum auf  Ma\-trix/\-Vektor-\-Multiplikationen beruhen.\vspace{0.5cm}\\
Alle mathematischen Grundkenntnisse, welche in diesen und weiteren Bereichen benötigt werden, können im Studium erlernt werden.