\subsection{Warum braucht man Mathematik in Informatik?}

Vor Allem in den ersten Semestern stehen die Mathematikvorlesungen im Informatik (Scientific Computing, Wirtschaftsinformatik, Geotelematik) immer etwas Abseits. Vielen Studierenden ist nicht klar, warum Mathematik so grundlegend wichtig f"ur ihr Studium ist - schlie\ss lich studieren sie ja keine (reine) Mathematik! Dieses Kapitel soll dieses Verständnisproblem l"osen und aufzeigen, welche Schnittpunkte zwischen Mathematik und Informatik bestehen. Es ist nämlich keineswegs so, dass Mathematik immer die graue, langweilige Theorie ist, die ``durchgestanden werden muss´´! Mit Mathematik haben Informatiker ein zusätzliches, sehr mächtiges Werkzeug zur Hand, mit dem viele scheinbar unl"osbare Probleme (in der Informatik!) pl"otzlich l"osbar werden. Das tolle ist: man muss nicht erst die komplette Mathematik verstanden haben, um sie als Werkzeug verwenden zu k"onnen. Es gilt eher der Grundsatz: mehr hilft mehr! In manchen Bereichen ist allerdings ein Grundverständnis notwendig, um sich das Werkzeug auch wirklich zu Nutze machen zu k"onnen. Einige dieser Bereiche versucht dieses Kapitel anzusprechen.

Zumindest grundlegende Mathematikkenntnisse sind in den folgenden Informatikbereichen wichtig. Diese Liste ist nicht vollst"andig und gibt nur einen kleinen Ausschnitt wider:

\begin{itemize}
\item Spieleentwicklung
\item Physiksimulationen
\item  Supercomputing
\item Robotics
\item Business Analytics
\item Data Mining
\item Machine Learning
\end{itemize}
 

Die Themen Spieleentwicklung, Business Analytics und Machine Learning werden im folgenden Text als Beispiele verwendet, dass Mathematik auch als Werkzeug f"ur Informatiker sehr n"utzlich ist.

\subsubsection{Spieleentwicklung}
Die Entwicklung eines Spiels umfasst neben Informatik noch sehr viele andere Bereiche: Design, Architektur, Psychologie, Soziologie, Teamf"uhrung, Finanzierung, Marketing und Musik sind alle in mehr oder weniger gro\ss em Umfang f"ur ein gutes Spiel n"otig. Die Mathematik ebenfalls: sobald sich ein Spielelement zumindest halbwegs realistisch bewegen soll, m"ussen die entsprechenden Bewegungsgleichungen (numerisch) integriert werden. Auch Lichteffekte wie Spiegelungen oder der Soundeffekte wie Widerhall m"ussen zuerst korrekt mathematisch beschrieben und dann implementiert werden, wenn sie "echt" aussehen sollen. Daneben sind auch Performanceprobleme zu l"osen: zum Beispiel wurde im Spiel Quake eine schnelle Wurzelberechnung verwendet, ohne die alles unspielbar langsam geworden w"are. (Zur Implementierung siehe \url{http://www5.in.tum.de/lehre/vorlesungen/konkr_math/SS_15/prog/NumPro_SS15_Programmieraufgabe_1.pdf})
Alle beschriebenen Probleme k"onnen mit relativ einfachen, mathematischen Methoden gel"ost werden - diese k"onnen alle im Studium gelernt werden (Analysis, Lineare Algebra, Numerik, Integraltransformationen, Differentialgleichungen).

\subsubsection{Business Analytics}
Der Begriff ``Big Data´´ ist in den letzten Jahren immer gebr"auchlicher geworden - Unternehmen versuchen aus gro\ss en Datenmengen, die sie "uber ihre Kunden sammeln, Informationen und Wissen zu generieren. Dieses Wissen soll dann in Verbesserung von Produkten oder in neue Produkte umgesetzt werden, die verkauft werden k"onnen. Business Analytics ist ein Teilgebiet von Data Mining, das speziell die Probleme in der Unternehmenswelt mittels Datenanalyse zu l"osen versucht. Solche Probleme sind zum Beispiel die Ermittlung von ``Risikokunden´´ von Handyvertr"agen, die unzufrieden sind und deshalb k"undigen wollen. K"onnen diese Kunden per Business Analytics gezielt ermittelt werden, werden ihnen Sonderangebote wie kostenlose Freiminuten geschenkt - mit dem Ziel, dass sie den Vertrag weiterlaufen lassen. Die Ermittlung der Kunden erfolgt durch statistische Analysen wie Regression oder Decision Trees - also alles wieder mathematische Werkzeuge, die im Studium erlernt werden k"onnen (Lineare Algebra, Wahrscheinlichkeitstheorie und Statistik).


\subsubsection{Machine Learning}
Durch die immer schneller rechnenden Computer k"onnen heute Probleme gel"ost werden, die fr"uher nur Menschen erledigen konnten. Dazu geh"oren automatische Gesichtserkennung, das Auffinden von Tumoren in medizinischen Bildern oder die Erkennung von Kreditkartenbetrug. Meist steht hinter diesen Anwendungen das sogenannte ``Machine Learning´´, was bedeutet, dass die Maschine (der Computer) selbst lernt, welche Eigenschaften in den auftretenden Daten (Bilder von Gesichtern, Kreditkarten"uberweisungen, Bilder von Tumoren oder gesundem Gewebe) zu welchem Ergebnis passt (Gesichter werden bestimmten Menschen zugeordnet, eine "uberweisung wird als Betrug markiert, ein Gewebebild wird als gesund gekennzeichnet). Das praktische am selbstlernenden Computer: der Mensch muss nur noch eine sogenannte Trainingsmenge an Datenmaterial bereitstellen, an der der Computer lernen kann - nach dem Training ist die Maschine selbst"andig in der Lage, eine Zuordnung zu machen - es ist also nicht n"otig, in m"uhevoller Kleinarbeit selbst die Eigenschaften eines Tumors oder eines Kreditkartenbetrugs herauszufinden und sie die entsprechende Software zu schreiben.
Die selbstlernenden Programme sind allerdings selbst von Menschen geschrieben und enthalten teilweise viel, teilweise aber auch relativ wenig Mathematik - es gibt allerdings keins, das komplett ohne mathematische Grundlagen in der Praxis gut funktioniert. Die sogenannten ``neuronalen Netze´´ (\url{https://en.wikipedia.org/wiki/Artificial_neural_network}) funktionieren im Hintergrund mit Matrix/Vektor-Multiplikationen, die schon im ersten Semester in Linearer Algebra gelernt werden k"onnen. Weitere Techniken wie Regression, Bildverarbeitung oder Filterungen k"onnen in weiterf"uhrenden, mathematischen F"achern erlernt werden und setzen auf die Grundlagen in Analysis (Integration, Differentiation, Funktionen) und Linearer Algebra (Matrizenrechnung) auf.