Beim Ableiten m"ochten wir etwas "uber die Ver"anderung einer Funktion erfahren. Angenommen ein Auto startet an einem Punkt und die Funktion $s(t)$ gibt an wie viele Meter das Auto zum Zeitpunkt $t$ zur"uckgelegt hat. Wir fragen uns nun wie schnell das Auto nach 10 Sekunden f"ahrt. Wie k"onnen wir dies herausfinden? Wir k"onnen das absch"atzen indem wir 
\begin{equation*}
\bar{v} = \frac{\text{Strecke}}{\text{Zeit}} \approx \frac{s(11) - s(9)}{11 - 9} = \frac{\Delta s}{\Delta t} = \frac{s(11) - s(9)}{2}
\end{equation*}
Aber ist $\bar{v} \stackrel{?}{=} v(10)$? Nein $\bar{v}$ ist nur die durchschnittliche Geschwindigkeit zwischen $s(9)$ und $s(11)$. Wenn wir die Geschwindigkeit genauer bestimmen m"ochten, m"ussen wir den Abstand/die \textit{Umgebung} von/um $10$ kleiner w"ahlen, $0$ darf er jedoch nicht werden, denn wir d"urfen nicht durch Null teilen. Hilfe gibt uns der Grenzwert:
\begin{equation*}
v(10) = \lim\limits_{h \to 0} \frac{s(10+h) - s(10)}{h}
\end{equation*}
Der Grenzwert muss nat"urlich existieren! Nehmen wir einmal an das Auto beschleunigt und $s(t) = 2 t^2$.
\begin{align*}
v(10) &= \lim\limits_{h \to 0} \frac{s(10+h) - s(10)}{h} = \lim\limits_{h \to 0} \frac{2 (10+h)^2 - 2 (10)^2}{h}\\
&= \lim\limits_{h \to 0} \frac{40h + 2h^2}{h} = \lim\limits_{h \to 0}(40 + 2h) \to 40
\end{align*}

\subsection{Ableiten von Funktionen}
Die Ableitung der Funktion $f$ an der Stelle $x_0$, geschrieben $f'(x_0)$ beschreibt das Verhalten der Funktion in der \textit{Umgebung} an der Stelle $x_0$. Sie ist auch die Steigung der Funktion an dem Punkt $x_0$. Die Tangente an $x_0$ hat die Steigung $f'(x_0)$. Eine Tangente ist eine Gerade, die den Graphen in nur einem einzigen Punkt ber"uhrt. Um die Tangentensteigung zu erhalten, betrachtet man zun"achst die Steigung einer Sekante und n"ahert den zweiten Punkt dem ersten immer weiter an.
\begin{definition}[Differenzierbarkeit in $x_0$]
Eine Funktion hei"st differenzierbar an der Stelle $x_0$ wenn der Grenzwert
\begin{equation*}
 \lim\limits_{x \to x_0} \frac{f(x)-f(x_0)}{x-x_0} = \lim\limits_{h \to 0}  \frac{f(x_0 + h)-f(x_0)}{h} , \text{ mit } h = x - x_0
\end{equation*}
existiert. Dieser Grenzwert wird Ableitung nach $x$ an der Stelle $x_0$ genannt und wird als $f'(x_0)$ oder $\frac{df}{dx}(x_0)$ notiert.
\end{definition}
Ein Grenzwert $a$ an der Stelle $x_0$ der Funktion $f$ existiert wenn der linke Grenzwert gleich dem rechten Grenzwert gleich dem Funktionswert gleich $a$ ist, also
\begin{equation*}
 \lim\limits_{x \to x_0^+} \frac{f(x)-f(x_0)}{x-x_0} =  \lim\limits_{x \to x_0^-} \frac{f(x)-f(x_0)}{x-x_0} = f(x_0) = a
\end{equation*}

\begin{definition}[Differenzierbarkeit]
Eine Funktion hei"st differenzierbar, wenn sie an jeder Stelle $x_0 \in \mathbb{D}$ differenzierbar ist.
\end{definition}

\paragraph{Anmerkung:} Eine differenzierbare Funktion ist stetig und eine in $x_0$ differenzierbare Funktion ist stetig in $x_0$. Die Umkehrung gilt nicht!

\subsubsection{Die erste Ableitung}
F"ur die gew"ohnlichen differenzierbaren Funktionen gibt es die uns bekannten Ableitungsregeln. Um eine Funktion abzuleiten multipliziert man bei Polynomen jeweils die Exponenten mit den dazugeh"origen Koeffizienten ihrer Basis und subtrahiert den Exponenten dabei um 1. Konstanten fallen dabei weg.
\begin{equation*}
f(x) = a_1 \cdot (x-a_2)^{a_3} + a_4 \Rightarrow f'(x) = a_1 \cdot (x - a_2)^{a_3-1} \cdot a_3 \text{ mit } a_1,a_2,a_3 \text{ konstant.}
\end{equation*}

\paragraph{Beispiel:}
\begin{equation*}
f(x)=2x^2 \Rightarrow f'(x)= 2 \cdot 2x^{2-1}=4x
\end{equation*}

\subsection{Besondere Ableitungen}
F"ur Logarithmus- und $e$-Funktionen sowie die Winkelfunktionen gelten besondere Ableitungsgesetze:
\begin{itemize}
\item $\sin'(x) = \cos(x)$
\item $\cos'(x) = - \sin(x)$
\item $\tan'(x) = \frac{1}{\cos^2(x)}$
\item $\ln'(x) = \frac{1}{x}$
\item $(e^x)' = e^x$
\end{itemize}

\subsection{Ableitungsregeln}
Wie "uberall in der Mathematik gibt es auch f"ur das Ableiten, insbesondere bei komplizierteren Funktionstermen, bestimmte Regeln im Hinblick auf ihre richtige Ableitung.

\subsubsection{Produktregel}
\begin{equation*}
f(x) = h(x) \cdot g(x) \Rightarrow f'(x)=h'(x) \cdot g(x)+h(x) \cdot g'(x)
\end{equation*}

\paragraph{Beispiel:}
\begin{equation*}
f(x)=2x^3 \cdot \sin(x)\Rightarrow f'(x)=6x^2 \cdot \sin(x)+2x^3 \cdot \cos(x)
\end{equation*}

\subsubsection{Quotientenregel}
\begin{equation*}
f(x)=\frac{h(x)}{g(x)} \Rightarrow f'(x)=\frac{h'(x) \cdot g(x)-h(x) \cdot g'(x)}{(g(x))^2}
\end{equation*}
Diese Regel kann auf die Produktregel zur"uckgef"uhrt werden man bemerke $\frac{h(x)}{g(x)} = h(x) \cdot g(x)^{-1}$

\paragraph{Beispiel:}
\begin{equation*}
f(x)=\frac{2x^3}{2x^2+4} \Rightarrow f'(x)=\frac{6x^2 \cdot (2x^2+4)-2x^3 \cdot 4x}{(2^2+4)^2}
\end{equation*}

\subsubsection{Kettenregel}
F"ur ineinander geschachtelte (verkettete) Funktionen gilt die sogenannte Kettenregel. Diese ist besonders n"utzlich und wird oft mehrmals hintereinander angewendet! Dabei wird die Funktion ganz normal abgeleitet und die innere Funktion \textcolor{red}{nachdifferenziert}.
\begin{equation*}
f(x)=f(g(x))\Rightarrow f'(g(x))=f'(g(x))\textcolor{red}{g'(x)}
\end{equation*}

\paragraph{Beispiel:}
\begin{equation*}
f(g(x))=x^{(x^2+4)}\Rightarrow f'(g(x))=(x^2+4)*x^{(x^2+4-1)}\textcolor{red}{ \cdot 2x}
\end{equation*}

\subsection{"Ubungen}
\begin{enumerate}
\item Bilde die Ableitungen folgender Terme
\begin{itemize}
\item $f(x)=12x^3 - 3x^2 + x + 12$
\item $f(x)= 6 \sin^2(x) + \cos(x) $
\item $f(x)= \frac{x^2-x}{x^2+x}$
\item $f(x)= \frac{3x}{12x^2 - 4x + 2} \cdot \sin^2(x) - 3 \cos(x) + 1$
\end{itemize}
Welche dieser Funktionen sind nicht differenzierbar?
\begin{itemize}
\item $f(x)=\lbrace^{3x; f"ur x < 3}_{2x; f"ur x>3}$
\item $f(x)=|x|$
\item $f(x)= \frac{x^3}{\sin(x)}$
\item $f(x)= 0$
\end{itemize}
\end{enumerate}