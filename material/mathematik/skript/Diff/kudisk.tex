\subsection{Kurvendiskussion}
Definition: Bei einer Kurvendiskussion wird der Graph einer Funktion im Hinblick auf seine Eigenschaften, wie seinen \textcolor{red}{Definitionsbereich}, \textcolor{red}{Grenzwerte}, \textcolor{red}{Asymptoten}, \textcolor{red}{Schnittpunkte mit den Koordinatenachsen}, \textcolor{red}{Symmetrie}, \textcolor{red}{Extrem- und Terrassenpunkte}, \textcolor{red}{Monotonie} und seinem \textcolor{red}{Kr"ummungsverhalten}, anhand des Funktionsterms untersucht.
\subsubsection{Definitonsbereich}
Zunächst beschreibt man, in welchem Bereich die Funktion definiert ist, indem man seine Definitions- und Wertemenge bestimmt.
\subsubsection{Schnittpunkte mit den Koordinatenachsen}
Interessant k"onnen auch die Schnittpunkte des Graphen mit den Koordinatenachsen sein, v.a. wenn man den Graphen skizzieren m"ochte.\\
\paragraph{Schnittpunkt mit der y-Achse}\hspace{2 cm}\\
Diesen erh"alt man, wenn man f"ur x den Wert 0 einsetzt.
\paragraph{Nullstellen}\hspace{2 cm}\\
Als n"achstes untersucht man die Funktion auf Nullstellen. Diese erh"alt man, indem man den Funktionsterm gleich Null setzt und nach x aufl"ost.\\
Merke: Ein Polynom hat h"ochstens so viele Nullstellen wie der h"ochste Exponent der Funktion, dabei erh"alt man zum Teil auch mehrfache Nullstellen, wie beispielsweise bei $x^3$. Diese Funktion hat ein dreifache Nullstelle an der Stelle x=0.
\subsubsection{Grenzwerte}
Nun betrachtet man das Verhalten des Graphen im Unendlichen. Dieses gibt man mit Hilfe des Limes an. Dabei ist interessant, ob sich der Graph einem bestimmten Wert ann"ahert oder ob er ins Unendliche steigt oder f"allt.
\subsubsection{Asymptoten}
 Asymptoten k"onnen Geraden aber auch andere Funktionen sein, an die sich der Graph der zu untersuchenden Funktion beliebig nah ann"ahert, ohne sie jemals zu ber"uhren.\\
 Asymptoten findet man vor allem bei gebrochen rationalen Funktionen. Dort, wo der Nenner den Wert 0 erreichen w"urde (an dieser Stelle ist die Funktion nicht definiert $=>$ Grenzwertbildung), befindet sich eine senkrechte Asymptote. Diese Stelle nennt man Polstelle. Es kann aber auch sein, dass es sich dort um ein Loch handelt, dies ist dann der Fall, wenn die Polstelle gleichsam eine Nullstelle der Funktion ist.\\
 Beispiel: $\frac{2x}{4x-1}$ Der Nenner wird 0 bei x=0.25, es handelt sich dabei um keine Nullstelle, somit befindet sich an der Stelle x=0,25 eine senkrechte Asymptote.\\
 Um den Graphen auf waagrechte oder andere Asymptoten zu untersuchen, schaut man sich die Exponenten der Variablen an.\\
 Die folgenden Ausf"uhrungen beziehen sich rein auf die h"ochsten Exponenten der Variablen im Z"ahler und Nenner.
 \subsubsection{Symmetrie}
 Hier untersuchen wir die Symmetrie die Graphen zum Koordinatensystem.\\
 Wir unterscheiden dabei Punktsymmetrie zum Koordinatenursprung und Achsensymmetrie zur y-Achse.\\
 Dabei untersucht man den Funktionsterm auf folgende Weise:
 \begin{itemize}
 \item Achsensymmetrie (y-Achse), wenn $f(x)=f(-x)$
 \item Punktsymmetrie (Ursprung), wenn $f(-x)=-f(x)$
 \end{itemize}
 \subsubsection{Extrem- und Terrassenpunkte}
  F"ur die Untersuchung des Funktionsgraphen ben"otigt man zun"achst die \textcolor{red}{erste Ableitung} des Funktionsterms. Bei einem Extrempunkt findet ein Vorzeichenwechsel der Steigung statt, d.h. die Ableitung ergibt in diesem Punkt 0. Bei einem Terrassenpunkt ist die Ableitung ebenfalls 0. Folglich m"ussen wir, um diese Punkte zu erhalten, die erste Ableitung gleich Null setzen.\\
  Somit ist die Nullstelle der ersten Ableitung entweder ein Extrem- oder Terrassenpunkt.\\
  Ob es sich um einen Extrempunkt oder einen Terrassenpunkt handelt, erfahren wir mit Hilfe der zweiten Ableitung.
  Wenn die Werte der zweiten Ableitung ungleich 0 sind, handelt es sich um einen Extrempunkt.\\
  Ist die zweite Ableitung an dieser Stelle jedoch 0, so wissen wir weiterhin nicht, ob es sich um einen Extrem- oder Terrassenpunkt handelt.\\
  Dabei gilt: 
  \begin{itemize}
  \item ist die zweite Ableitung negativ, handelt es sich um ein lokales Maximum
  \item ist die zweite Ableitung positiv, handelt es sich um ein lokales Minimum
  \end{itemize}
  Kurz: ein lokales Maximum oder Minimum besteht immer dann, wenn die erste Ableitung an ihrer Nullstelle einen Vorzeichenwechsel hat. Andernfalls handelt es sich um einen Terrassenpunkt.
 \subsubsection{Monotonie}
  Nachdem wir die Funktion hinreichend auf Extrem- und Terrassenpunkte untersucht haben, kann man ihre Monotonie analysieren. Diese gibt man "ublicherweise in Intervallen an, deren Grenzen die Ränder und die Extremstellen bilden.
  \subsubsection{Kr"ummung}
 Um das Kr"ummungsverhalten einer Funktion zu untersuchen, ben"otigt man die zweite Ableitung des Funktionsterms.\\
 Es gilt: $f''(x)<0 \Rightarrow \textcolor{red}{rechtsgekr"ummt}$\\
 \hspace{1.4 cm}$f''(x)>0 \Rightarrow \textcolor{red}{linksgekr"ummt}$\\
 Die Stelle, an welcher der Graph seine Kr"ummung "andert, bezeichnet man als Wendestelle.\\
 Diese ist gleich der Nullstelle der zweiten Ableitung des Funktionsterms, d.h. die zweite Ableitung gleich 0 setzten und den zugeh"origen x-Wert berechnen.\vspace{0.5 cm}\\
 Das sind die wesentlichen Schritte einer Kurvendiskussion. Die Reihenfolge der Bearbeitung kann dabei nat"urlich variiert werden. Punkte, an denen sich die Richtung der Kr"ummung von rechts nach links, bzw. von links nach rechts "andert, nennt man Wendepunkte.
 \subsection{"Ubungen}
 \begin{enumerate}
 \item Diskutiere die Kurven folgender Funktionen
 \begin{itemize}
 \item $f(x)=x^4$
 \item $f(x)=sin(x)$
 \item $f(x)=3x^2 + 12x - 4$
 \item $f(x)=\frac{4x^3 + x^2 - x}{x+1}$
 \item $f(x)=\frac{tan^2(x)}{sin^2(x)}$
 \end{itemize}
 \item Untersuche folgende Funktionen auf ihre Nullstellen, Extrempunkte und Kr"ummung.
 \begin{itemize}
 \item $f(x)=(x-4)^2-3$
 \item $f(x)=(x-2)(x+5)(x-4)$
 \item $f(x)=\frac{1}{x}$
 \end{itemize}
 \end{enumerate}
 
 