\subsection{Definition}
Eine \textbf{Funktion} (oder \textbf{Abbildung}) $f$ ist eine Relation zwischen zwei Mengen, die \textcolor{red}{jedem} Element der einen Menge (Funktionsargument $x$) \textcolor{red}{genau ein} Element der anderen Menge (Funktionswert $y$) zuordnet, so dass $$y=f(x).$$
Ist $x_1 \neq x_2$, dann ist $f(x_1) = f(x_2)$ nat"urlich zul"assig.
\begin{enumerate}
\item Die \textbf{Definitionsmenge} $\mathbb{D}$ von $f$ ist die Menge, aus der das Funktionsargument $x$ stammt.
\item Die \textbf{Zielmenge} $Z$ von $f$ ist die Menge der Funktionswerte $y$.
\item Die \textbf{Wertemenge} oder besser der \textbf{Bildbereich} von $f$ (manchmal auch nur das \textbf{Bild} von $f$) $\mathbb{W} \subset Z$ ist die Menge der Funktionswerte, die die Funktion annehmen kann. Oder kurz
\begin{equation*}
\mathbb{W} := \left\{ f(x) \ | x \in \mathbb{D} \right\}
\end{equation*}
\end{enumerate}
\begin{flushleft}
Wir schreiben kurz
\begin{equation*}
f : \mathbb{D} \to Z, x \mapsto f(x) \text{ oder } f : \mathbb{D} \to Z, \text{\glqq}f(x)\text{\grqq} := f(x)
\end{equation*}
um anzugeben was $\mathbb{D}$ und $Z$ ist und wie die Funktion aussieht, also zum Beispiel:
\begin{equation*}
f :\mathbb{R} \to \mathbb{R}, x \mapsto 2x + 5 \text{ bzw. } f(x) := 2x + 5
\end{equation*}
\begin{warning}Jedes $x \in \mathbb{D}$ hat sein $y \in Z$, aber nicht jedes $y \in Z$ sein $x \in \mathbb{D}$ (sprich jedem Wert der Definitionsmenge wird ein Zielwert zugeordnet, aber nicht jeder Zielwert muss einen Wert in der Definitionsmenge haben). Jedoch hat jedes $y \in \mathbb{W}$ sein $x \in \mathbb{D}$ (jeder Wert im Bildbereich der Funktion hat einen zugeh"origen Wert im Definitionsbereich).
\end{warning}
Die Definitions\-menge gibt hierbei an, in welchem Bereich die Funktion definiert ist, sprich welche Werte \glqq in die Funktion eingesetzt werden d"urfen\grqq . Die Abbildungs\-vorschrift sorgt nun daf"ur, dass jedem Wert $x$ des Definitions\-bereichs genau ein Wert $y$ des Bild\-bereiches zugeordnet werden kann. Die Wertemenge enth"alt genau jene Zahlen, welche man durch Abbildung des Definitions\-bereiches mithilfe der Abbildungs\-vorschrift erhalten kann.

\paragraph{Zur Vorstellungserleichterung:}
Vorstellen kann man sich das ganze wie die Erschaffung eines Gem"aldes. Die Definitionsmenge ist hierbei die abzubildende Szene (zum Beispiel eine Landschaft).\\
Die Zielmenge sind alle Farben, welche der K"unstler besitzt (und mischen kann), um das Bild anzufertigen.\\
Die Wertemenge, ist die Menge der Farben, welche der K"unstler wirklich verwendet, um das Gem"alde zu erzeugen.

\begin{warning}
	Die Worte Bild und Urbild, stammen tats"achlich aus dem Bereich der Kunst. Das Urbild ist die Originalszene und das Bild die Abbildung dieser Szene, welche der K"unstler erstellt. In der Mathematik werden die x-Werte ab und an auch als Urbild und die y-Werte als Bild bezeichnet.
\end{warning}

\begin{figure}[h!]
 \centering
 \begin{tikzpicture}[ele/.style={fill=black,circle,minimum width=.8pt,inner sep=1pt},every fit/.style={ellipse,draw,inner sep=-2pt}, scale = 0.8]
  \node[ele,label=left:$a$] (a1) at (0,4) {};    
  \node[ele,label=left:$b$] (a2) at (0,3) {};    
  \node[ele,label=left:$c$] (a3) at (0,2) {};
  \node[ele,label=left:$d$] (a4) at (0,1) {};

  \node[ele,,label=right:$1$] (b1) at (4,5) {};
  \node[ele,,label=right:$2$] (b2) at (4,4) {};
  \node[ele,,label=right:$3$] (b3) at (4,3) {};
  \node[ele,,label=right:$4$] (b4) at (4,2) {};
  \node[ele,,label=right:$5$] (b5) at (4,1) {};  
  

  \node[draw,fit= (a1) (a2) (a3) (a4),minimum width=2cm] {} ;
  \node[draw,fit= (b1) (b2) (b3) (b4) (b5),minimum width=2cm] {} ;  
  \draw[->,thick,shorten <=2pt,shorten >=2pt] (a1) -- (b4);
  \draw[->,thick,shorten <=2pt,shorten >=2] (a2) -- (b2);
  \draw[->,thick,shorten <=2pt,shorten >=2] (a3) -- (b1);
  \draw[->,thick,shorten <=2pt,shorten >=2] (a4) -- (b4);
 \end{tikzpicture}
 \caption{Eine Funktion mit einer endlichen Definitionsmenge $\mathbb{D} = \{a, b, c, d\}$  (links), einer endlichen Zielmenge $Z = \{1, 2, 3, 4, 5\}$ (rechts) und einer Bildmenge $\mathbb{W} = \{1,2,4\}$. Alle Elemente aus $\mathbb{D}$ besitzen genau eine \glqq Verbindung\grqq .}
\end{figure}

\paragraph{Beispiele:}
\begin{itemize}
\item $\mathbb{D}\rightarrow\mathbb{W}, f(x) := x^2$ bzw. $x \mapsto x^2$
\item $\mathbb{D}\rightarrow\mathbb{W}, f(x) := 42x(12x-12(3x^2))$ bzw. $x \mapsto 42x(12x - 12(3x^2))$
\end{itemize}
\end{flushleft}

\subsection{Grundlegende Eigenschaften von Funktionen (Bonus)}
\subsubsection{Surjektivit"at (Bonus)}
Eine Funktion $f$ ist surjektiv (siehe Abb. \ref{fig:surjektiv}) genau dann wenn jeder Wert der Zielmenge mindestens einmal angenommen wird:
\begin{equation*}
\forall y \in Z \ \exists x \in \mathbb{D} : f(x) = y
\end{equation*}
\begin{figure}[h!]
 \centering
 \begin{tikzpicture}[ele/.style={fill=black,circle,minimum width=.8pt,inner sep=1pt},every fit/.style={ellipse,draw,inner sep=-2pt}, scale = 0.8]
  \node[ele,label=left:$a$] (a1) at (0,6) {};    
  \node[ele,label=left:$b$] (a2) at (0,5) {};    
  \node[ele,label=left:$c$] (a3) at (0,4) {};
  \node[ele,label=left:$d$] (a4) at (0,3) {};
  \node[ele,label=left:$e$] (a5) at (0,2) {};
  \node[ele,label=left:$f$] (a6) at (0,1) {};

  \node[ele,,label=right:$1$] (b1) at (4,5) {};
  \node[ele,,label=right:$2$] (b2) at (4,4) {};
  \node[ele,,label=right:$3$] (b3) at (4,3) {};
  \node[ele,,label=right:$4$] (b4) at (4,2) {};
  \node[ele,,label=right:$5$] (b5) at (4,1) {};  
  

  \node[draw,fit= (a1) (a2) (a3) (a4) (a5) (a6),minimum width=2cm] {} ;
  \node[draw,fit= (b1) (b2) (b3) (b4) (b5),minimum width=2cm] {} ;  
  \draw[->,thick,shorten <=2pt,shorten >=2pt] (a1) -- (b4);
  \draw[->,thick,shorten <=2pt,shorten >=2] (a2) -- (b2);
  \draw[->,thick,shorten <=2pt,shorten >=2] (a3) -- (b1);
  \draw[->,thick,shorten <=2pt,shorten >=2] (a4) -- (b4);
  \draw[->,thick,shorten <=2pt,shorten >=2] (a5) -- (b3);
  \draw[->,thick,shorten <=2pt,shorten >=2] (a6) -- (b5);
 \end{tikzpicture} 
 \caption{Eine surjektive Funktion.}
 \label{fig:surjektiv}
\end{figure}

\subsubsection{Injektivit"at (Bonus)}
Eine Funktion $f$ ist injektiv (siehe Abb. \ref{fig:injektiv}) genau dann wenn jeder Wert der Bildmenge \textbf{nicht mehrmals} angenommen wird:
\begin{equation*}
\forall x_1, x_2 \in \mathbb{D} : f(x_1) = f(x_2) \Rightarrow x_1 = x_2
\end{equation*}
\begin{figure}[h!]
 \centering
 \begin{tikzpicture}[ele/.style={fill=black,circle,minimum width=.8pt,inner sep=1pt},every fit/.style={ellipse,draw,inner sep=-2pt}, scale = 0.8]
  \node[ele,label=left:$a$] (a1) at (0,4) {};    
  \node[ele,label=left:$b$] (a2) at (0,3) {};    
  \node[ele,label=left:$c$] (a3) at (0,2) {};
  \node[ele,label=left:$d$] (a4) at (0,1) {};

  \node[ele,,label=right:$1$] (b1) at (4,5) {};
  \node[ele,,label=right:$2$] (b2) at (4,4) {};
  \node[ele,,label=right:$3$] (b3) at (4,3) {};
  \node[ele,,label=right:$4$] (b4) at (4,2) {};
  \node[ele,,label=right:$5$] (b5) at (4,1) {};  
  

  \node[draw,fit= (a1) (a2) (a3) (a4) (a5) (a6),minimum width=2cm] {} ;
  \node[draw,fit= (b1) (b2) (b3) (b4) (b5),minimum width=2cm] {} ;  
  \draw[->,thick,shorten <=2pt,shorten >=2pt] (a1) -- (b1);
  \draw[->,thick,shorten <=2pt,shorten >=2] (a2) -- (b2);
  \draw[->,thick,shorten <=2pt,shorten >=2] (a3) -- (b3);
  \draw[->,thick,shorten <=2pt,shorten >=2] (a4) -- (b4);
 \end{tikzpicture} 
 \caption{Eine injektive Funktion.}
 \label{fig:injektiv}
\end{figure}

\subsubsection{Bijektion (Bonus)}
Eine Funktion ist bijektiv (siehe Abb. \ref{fig:bijectiv}), wenn sie zugleich injektiv und surjektiv ist. Eine solche Funktion ist eine eins-zu-eins Relation, sie besitzt zudem eine Umkehrfunktion.
\begin{figure}[h!]
 \centering
 \begin{tikzpicture}[ele/.style={fill=black,circle,minimum width=.8pt,inner sep=1pt},every fit/.style={ellipse,draw,inner sep=-2pt}, scale = 0.8]
  \node[ele,label=left:$a$] (a1) at (0,4) {};    
  \node[ele,label=left:$b$] (a2) at (0,3) {};    
  \node[ele,label=left:$c$] (a3) at (0,2) {};
  \node[ele,label=left:$d$] (a4) at (0,1) {};

  \node[ele,,label=right:$1$] (b1) at (4,4) {};
  \node[ele,,label=right:$2$] (b2) at (4,3) {};
  \node[ele,,label=right:$3$] (b3) at (4,2) {};
  \node[ele,,label=right:$4$] (b4) at (4,1) {};
  

  \node[draw,fit= (a1) (a2) (a3) (a4) (a5) (a6),minimum width=2cm] {} ;
  \node[draw,fit= (b1) (b2) (b3) (b4) (b5),minimum width=2cm] {} ;  
  \draw[->,thick,shorten <=2pt,shorten >=2pt] (a1) -- (b1);
  \draw[->,thick,shorten <=2pt,shorten >=2] (a2) -- (b2);
  \draw[->,thick,shorten <=2pt,shorten >=2] (a3) -- (b3);
  \draw[->,thick,shorten <=2pt,shorten >=2] (a4) -- (b4);
 \end{tikzpicture} 
 \caption{Eine bijektive Funktion.}
 \label{fig:bijectiv}
\end{figure}
