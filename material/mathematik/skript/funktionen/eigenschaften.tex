\subsection{Grundlegende Eigenschaften von Funktionen}
\subsubsection{Was ist eine Funktion?}
Eine \textbf{Funktion} $f$ oder \textbf{Abbildung} $f$ ist eine Relation zwischen zwei Mengen, die \textcolor{red}{jedem} Element der einen Menge (Funktionsargument $x$) \textcolor{red}{genau ein} Element der anderen Menge (Funktionswert $y$) zuordnet.
\begin{enumerate}
\item Die \textbf{Definitionsmenge} $\mathbb{D}$ ist die Menge aus der das Funktionsargument $x$ stammt.
\item Die \textbf{Zielmenge} $Z$ ist die Menge der Funktionswerte $y$.
\item Die \textbf{Wertemenge} oder der besser \textbf{Bildbereich} $\mathbb{W} \subset Z$ ist die Menge der Funktionswerte die die Funktion annehmen kann
\end{enumerate}
\begin{flushleft}
kurz schreiben wir
\begin{equation*}
f : \mathbb{D} \to Z, x \mapsto y
\end{equation*}
Man beachte, dass zwar jedes $x \in \mathbb{D}$ sein $y \in Z$ hat aber nicht jedes $y \in Z$ sein $x \in \mathbb{D}$ (salopp ausgedr"uckt). Jedoch aht jedes $x \in \mathbb{W}$ sein $x \in \mathbb{D}$. Die Definitionsmenge gibt hierbei an in welchem Bereich die Funktion definiert ist, sprich welche Werte "'in die Funktion eingesetzt werden d"urfen"'. Die Abbildungsvorschrift sorgt nun daf"ur, dass jedem Wert $x$ des Definitionsbereichs genau ein Wert $y$ des Bildbereiches zugeordnet werden kann. Die Wertemenge enth"alt genau jene Zahlen, welche man durch Abbildung des Definitionsbereiches mithilfe der Abbildungsvorschrift erhalten kann.

\paragraph{Zur Vorstellungserleichterung:}
Vorstellen kann man sich das wie die Funktionsweise eines Fotoapparates. Der Definitionsbereich w"are hierbei das, was in der realen Welt existiert und von der Kamera aufgenommen werden soll. Das ausgestrahlte Licht, der Bestandteile des Definitionsbereiches, wird nun mithilfe einer Linse eingefangen, gespiegelt und bei neueren Kameras mithilfe von Sensoren in digitale Bilder umgewandelt. Dieses digitale \textcolor{red}{Bild} ist nun das Ergebnis und jeder erreichte Wert (RGB, CMYK, ...) ist Bestandteil der Bildmenge.

\paragraph{Beispiele:}
\begin{itemize}
\item $\mathbb{D}\rightarrow\mathbb{W}, f(x)=x^2$
\item $\mathbb{D}\rightarrow\mathbb{W}, f(x)=42x(12x-12(3x^2))$
\end{itemize}
\end{flushleft}

\subsubsection{Umkehrfunktion}
Durch Einsetzen eines bestimmten $x$-Wertes in eine Funktion $f(x)$ erh"alt man den entsprechenden $y$-Wert. Will man jedoch den $x$-Wert zu einem bestimmten $y$-Wert bestimmen, so muss man zun"achst dessen Umkehrfunktion $f(y)^{-1}$ berechnen. Daf"ur l"ost man die Funktion nach $x$ auf. Durch das Bilden der Umkehrfunktion werden Definitions- und Wertemenge vertauscht. Die Umkehrfunktion muss nat"urlich die Anforderungen erf"ullen welche an eine Funktion gestellt werden, erf"ullen. Wenn nicht jedes $y$ genau ein $x$ abgebildet wird existiert die Umkehrfunktion nicht. Eine Funktion muss \textbf{bijektiv} sein (jeder Wert der \textbf{Zielmenge} wird angenommen (\textcolor{red}{surjektiv}) und kein Wert der \textbf{Bildmenge} wird mehrfach angenommen (\textcolor{red}{injektiv})), damit aus ihr eine Umkehrfunktion gebildet werden kann.

\paragraph{Beispiel}
$f(x) = x^2$ besitzt keine Umkehrfunktion denn $f(2) = f(-2) = 4$. Somit m"usste $f^{-1}(4) = \{2, -2\}$ (hier wird ein Wert $4$ der Bildmenge mehrfach angenommen) sein, was aber die Definition einer Funktion widerspricht!

\subsubsection{Monotonie}
Wie bei Folgen und Reihen gibt es auch in der Welt der Funktionen den Begriff (strenge) Monotonie.\\
\begin{itemize}
\item Eine Funktion ist \textcolor{red}{monoton fallend} \textcolor{DarkGrey}{(bzw. wachsend)}, wenn gilt:\\
$\forall x_1, x_2 \in\mathbb{D}, x_1 < x_2: f(x_1)\geq f(x_2) \hspace{0.2 cm}\textcolor{DarkGrey}{(bzw. f(x_1)\leq f(x_2))}$
\item Eine Funktion ist \textcolor{red}{streng monoton fallend} \textcolor{DarkGrey}{(bzw. wachsend)}, wenn gilt:\\
$\forall x_1, x_2 \in\mathbb{D}, x_1 < x_2: f(x_1)>f(x_2) \hspace{0.2 cm}\textcolor{DarkGrey}{(bzw. f(x_1)<f(x_2))}$
\end{itemize}

\subsubsection{Symmetrie}
Eine Funktion ist genau dann \textcolor{red}{symmetrisch zur y-Achse}, wenn gilt:
\begin{equation*}
f(x)=f(-x)
\end{equation*}
Solche Funktionen werden auch als \textcolor{red}{gerade} Funktion bezeichnet. Eine Funktion ist genau dann \textcolor{red}{symmetrisch zum Ursprung}, oder auch \textcolor{red}{ungerade}, wenn gilt:
\begin{equation*}
f(-x)=-f(x)
\end{equation*}
\begin{center}
\textbf{\textcolor{red}{Achtung:}} Gerade hat in dem Sinne nichts mit der Geraden zu tun!!!
\end{center}

\subsubsection{Periodizit"at}
Eine Funktion ist dann \textcolor{red}{periodisch}, wenn es eine Konstante p gibt, f"ur die gilt:
\begin{equation*}
f(x+p) = f(x), \forall x\in\mathbb{D}
\end{equation*}
Diese Definition sagt eigentlich nichts anderes aus, als dass sich die Funktion st"andig im gleichen Abstand wiederholt.

\subsubsection{Nullstellen}
Eine Nullstelle ist ein Punkt, an dem der Graf die x-Achse schneidet. Anders ausgedr"uckt: Ein Punkt an dem die Funktion $f(x)$ den Wert $0$ erreicht. Um Nullstellen zu berechnen, l"ost man
\begin{equation*}
f(x) = 0,
\end{equation*}
indem die Gleichung nach $x$ aufgel"ost wird. So kann man nun den $x$-Wert berechnen.

\subsubsection{Stetigkeit}
Eine Funktion ist stetig, wenn man ihren Graphen (innerhalb der Definitionsmenge) ohne Absetzen des Stiftes in einem Zug zeichnen kann (salopp). Achtung folgende Definition ist eine der knackigsten Ausdr"ucke die euch begegnen wird. Eine Funktion $f : \mathbb{D} \to Z$ ist stetig in $x_0 \in \mathbb{D}$, wenn
\begin{equation*}
\forall \epsilon > 0 \ \exists \delta > 0 : \forall x \in \mathbb{D} \text{ mit } \left|x - x_0 \right| < \delta \text{ gilt: } \left| f(x) - f(x_0) \right| < \epsilon 
\end{equation*}
Intuitiv bedeutet dies, dass egal wie klein ich meine Funktionswert-Umgebung ($\epsilon$) um $f(x_0)$ w"ahle, ich trotzdem alle Funktionswerte welche durch die Argument-Umgebung ($\delta$) um $x_0$ entstehen einschlie"sen kann.

\subsubsection{Verschieben einer Funktion}
Angenommen wir haben eine Funktion $f$ mit $\mathbb{D} = \mathbb{R}$ und wollen eine neue Funktion $f'$ bauen, wobei wir diese um $d > 0$ nach rechts verschieben wollen. Damit muss gelten:
\begin{equation*}
\forall x \in \mathbb{R} :  f'(x) = f(x-d) \iff \forall x \in \mathbb{R} : f'(x+d) = f(x).
\end{equation*}
Wir nehmen also einfach die gegebene Funktion $f(x)$ und ersetzen jedes $x$ durch $x-d$. Setzen wir hingegen $f'(x) = f(x) + d$ so verschieben wir die Funktion nach oben. F"ur $d < 0$ folgt, dass wir die Funktion nach links bzw. nach unten verschieben.