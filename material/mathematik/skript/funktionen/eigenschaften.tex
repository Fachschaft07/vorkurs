\subsection{Grundlegende Eigenschaften von Funktionen}
\subsubsection{Was ist eine Funktion?}
Eine Funktion besteht grundlegend aus drei Dingen:
\begin{enumerate}
\item\textcolor{red}{Definitionsmenge:} $\mathbb{D}$
\item\textcolor{red}{Bildbereich/Wertemenge:} $\mathbb{W}$
\item\textcolor{red}{Abbildungsvorschrift:} $f$
\end{enumerate}
\begin{flushleft}
Die Definitionsmenge gibt hierbei an in welchem Bereich die Funktion definiert ist, sprich welche Werte "'in die Funktion eingesetzt werden d"urfen"'.\\\vspace{0.5 cm}
Die Abbildungsvorschrift sorgt nun daf"ur, dass jedem Wert x des Definitionsbereichs genau ein Wert y des Bildbereiches zugeordnet werden kann.\\\vspace{0.5 cm}
Die Wertemenge enth"alt genau jene Zahlen, welche man durch Abbildung des Definitionsbereiches mithilfe der Abbildungsvorschrift erhalten kann.\\\vspace{0.5 cm}
\textbf{Zur Vorstellungserleichterung:}\\
Vorstellen kann man sich das wie die Funktionsweise eines Fotoapparates.\\
Der Definitionsbereich w"are hierbei das, was in der realen Welt existiert und von der Kamera aufgenommen werden soll.
Das ausgestrahlte Licht, der Bestandteile des Definitionsbereiches, wird nun mithilfe einer Linse eingefangen, gespiegelt und bei neueren Kameras mithilfe von Sensoren in digitale Bilder umgewandelt. Dieses digitale \textcolor{red}{Bild} ist nun das Ergebnis und jeder erreichte Wert (RGB, CMYK, ...) ist Bestandteil der Bildmenge.\\\vspace{0.5 cm}
\textbf{Beispiele:}\\
\begin{itemize}
\item $\mathbb{D}\rightarrow\mathbb{W}, f(x)=x^2$
\item $\mathbb{D}\rightarrow\mathbb{W}, f(x)=42x(12x-12(3x^2))$
\end{itemize}
\end{flushleft}

\subsubsection{Monotonie}
Wie bei Folgen und Reihen gibt es auch in der Welt der Funktionen den Begriff (strenge) Monotonie.\\
\begin{itemize}
\item Eine Funktion ist \textcolor{red}{monoton fallend} \textcolor{DarkGrey}{(bzw. wachsend)}, wenn gilt:\\
$\forall x_1, x_2 \in\mathbb{D}, x_1 < x_2: f(x_1)\geq f(x_2) \hspace{0.2 cm}\textcolor{DarkGrey}{(bzw. f(x_1)\leq f(x_2)}$
\item Eine Funktion ist \textcolor{red}{streng monoton fallend} \textcolor{DarkGrey}{(bzw. wachsend)}, wenn gilt:\\
$\forall x_1, x_2 \in\mathbb{D}, x_1 < x_2: f(x_1)>f(x_2) \hspace{0.2 cm}\textcolor{DarkGrey}{(bzw. f(x_1)<f(x_2)}$
\end{itemize}

\subsubsection{Symmetrie}
\begin{flushleft}
Eine Funktion ist genau dann \textcolor{red}{symmetrisch zur y-Achse}, wenn gilt:\\
$f(x)=f(-x)$\\
Solche Funktionen werden auch als \textcolor{red}{gerade} Funktion bezeichnet.\\\vspace{0.5 cm}
Eine Funktion ist genau dann \textcolor{red}{symmetrisch zum Ursprung}, oder auch \textcolor{red}{ungerade}, wenn gilt:\\
$f(-x)=-f(x)$\\
\end{flushleft}
\textbf{\textcolor{red}{Achtung:}} Eine Gerade ist nicht immer eine gerade Funktion!!!

\subsubsection{Periodizit"at}
Eine Funktion ist dann \textcolor{red}{periodisch}, wenn es eine Konstante p gibt, f"ur die gilt:\\
$f(x+p) = f(x), \forall x\in\mathbb{D}$\vspace{0.5 cm}\\
Diese Definition sagt eigentlich nichts anderes aus, als dass sich die Funktion st"andig im gleichen Abstand wiederholt.

\subsubsection{Nullstellen}
Eine Nullstelle ist ein Punkt, an dem der Graf die x-Achse schneidet.\\
Anders ausgedr"uckt: Ein Punkt an dem die Funktion f(x) den Wert 0 erreicht.\\
Um Nullstellen zu berechnen, setzt man im Normalfall die Gleichung erst einmal gleich 0 und stellt diese dann nach "'x"' um. So kann man nun den x-Wert berechnen (der y-Wert eine Nullstelle ist logischerweise 0, weshalb er nicht berechnet werden muss).

\subsubsection{Stetigkeit}
Eine Funktion ist stetig, wenn man ihren Graphen (innerhalb der Definitionsmenge) ohne Absetzen des Stiftes in einem Zug zeichnen kann.