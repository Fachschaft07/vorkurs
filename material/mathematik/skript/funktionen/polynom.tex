\subsection{Polynome} \label{polynom}
\subsubsection{Was ist ein Polynom}
Bei Funktionen der Form $f(x) = a_n x^b + a_{n-1} x^{b-1} + a_{n-2} x^{b-2} + ... a_0$ handelt es sich um so genannte Polynome. Da sie sehr einfach differenzierbar und integrierbar sind, haben sie eine sehr wichtige Rolle als mathematisches Werkzeug, vor allem in der Numerik. Hier n"ahert man komplexe Funktionen mit Polynomen an und kann diese dann einfach ableiten oder integrieren, was mit der komplexen Funktion meist nicht so einfach ist.

\subsubsection{Nullstellenberechnung mithilfe der Polynomdivision (Bonus)}
Im Folgenden werde wir nun anhand des Beispiels 
\begin{equation*}
f(x) = 6x^3 - 12x^2 - 6x + 12
\end{equation*}
die Nullstellenberechnung mithilfe der Polynomdivision erl"autern. Dabei werden alle Nullstellen erraten. Jede Nullstelle liefert einen Linearfaktor. 
\begin{enumerate}
\item Wie immer setzen wir die Funktion gleich Null und werden den Faktor los:
\begin{equation*}
6x^3 - 12x^2 - 6x + 12 = 0 \iff x^3 -2x^2 - x + 2 = 0
\end{equation*}

\item Nun muss man (leider) die erste Nullstelle "'erraten"'. Da es aber sehr viele Zahlen gibt, w"urde dies ohne einen kleinen Trick sehr schwer werden. Besitzt ein Polynom lediglich ganzzahlige Koeffizienten, so sind die $x$-Werte der Nullstellen ganzzahlige Teiler von des letzten Koeffizienten ($a_0$). Warum dies so ist, schauen wir uns sp"ater an. Unser Beispiel kann also nur Nullstellen bei $x$-Werten aus $\{-1, 1, -2, 2\}$, besitzen. Nun kann man systematisch alle m"oglichen Kandidaten abarbeiten oder sieht es direkt. Testen mit $x \stackrel{?}{=} -1$:
\begin{equation*}
(-1)^3 - 2(-1)^2 - (-1) + 2 = 0
\end{equation*}
Gl"ucklicherweise haben wir schon beim ersten Versuch eine Nullstelle bei $x_1 = -1$ entdeckt.
\item Nun werden wir uns die Eigenschaft des Faktorisierens zu Nutze machen, dass jedes Polynom in Faktoren $(x-x_{Nullstelle})$ zerlegt werden kann. Wir werden nun eine Division des Polynoms durch $(x-(-1)) = (x + 1)$ durchf"uhren:
\polylongdiv{x^3 -2x^2 - x + 2}{x + 1}

\item Jetzt beginnt der Vorgang mit der neuen Funktion ($g(x) = x^2 - 3x + 2$), welche einen Grad weniger hat von Neuem. In unserem Fall jedoch, da wir nun eine quadratische Funktion haben, k"onnen wir die Mitternachtsformel zur L"osung heranziehen. Die drei Nullstellen unserer Beispielfunktion lauten:
\begin{equation*}
x_0 = -1; x_1 = 1; x_2 = 2
\end{equation*}
Die Funktion l"asst sich also auch wie folgt schreiben:
\begin{equation*}
f(x)=(x+1)(x-1)(x-2)
\end{equation*}
\end{enumerate}
Anhand der "'neuen Darstellungsart"' der Funktion können wir sehen, warum alle Nullstellen, unter der Voraussetzung, dass sie alle ganzzahlig sind, ganzzahlige Teiler des letzten Koeffizienten sind.\\
Nehmen wir daf"ur ein generisches Beispiel zur Hand:\\
$f(x)=(x-a)(x-b)(x-c)=(x^2 - ax - bx + ab)(x-c)=(x^3 - ax^2 - bx^2 + abx - cx^2 + acx + bcx - abc)$\\
Man kann also sehen, dass der Betrag des letzten Koeffizienten lediglich die Multiplikation aller Nullstellen ist. Sind diese nun alle ganzzahlig, so ist auch jede Nullstelle ein ganzzahliger Teiler des letzten Koeffizienten.