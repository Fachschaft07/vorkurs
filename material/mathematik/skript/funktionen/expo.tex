\subsection{Exponentialfunktionen}
\subsubsection{Was sind Exponentialfunktionen?}
Exponentialfunktionen sind Funktionen der Form 
\begin{equation*}
f(x)=a \cdot b^{x-x_0}+y_0
\end{equation*}
Dabei ist
\begin{description}
\item[$a$] Stauchungs- bzw. Streckungsfaktor
\item[$b$] Basis
\item[$x_0$] Wert der Verschiebung in $x$-Richtung (siehe Abschnitt \ref{sec:verschieben})
\item[$y_0$] Wert der Verschiebung in $y$-Richtung (siehe Abschnitt \ref{sec:verschieben})
\end{description}
Sie ist auf ganz $\mathbb{R}$ definiert und ist f"ur die Zielmenge $\mathbb{R}^{+} \setminus \{0\}$ surjektiv. Sie ist au"serdem injektiv uns somit f"ur $\mathbb{R}^{+} \setminus \{0\}$ als Zielmenge bijektiv.

\subsubsection{Umkehrfunktion - die Logarithmusfunktion}
Die Umkehrfunktion der Exponentialfunktion ist die Logarithmusfunktion. Sei $f(x)$ gleich
\begin{equation*}
f(x) = a \cdot b^{x-x_0}+y_0
\end{equation*}
So berechnet sich die Umkehrfunktion wie folgt:
\begin{align*}
y = a \cdot b^{x-x_0}+y_0 &\iff  \\
y - y_0 = a \cdot b^{x-x_0} &\iff \\
\frac{y - y_0}{a} =  b^{x-x_0} &\iff \\
\log_b\left(\frac{y - y_0}{a}\right) = x - x_0 &\iff \\
\log_b\left(\frac{y - y_0}{a}\right) + x_0 = x &
\end{align*}
Da die Exponentialfunktion mit einer Zielmenge gleich $\mathbb{R}^{+} \setminus \{0\}$ bijektiv ist, ist die Umkehrfunktion, der die Logarithmusfunktion, nur auf $\mathbb{R}^{+} \setminus \{0\}$ definiert!