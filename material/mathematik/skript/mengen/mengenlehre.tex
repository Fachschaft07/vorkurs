In diesem Kapitel möchten wir einen weiteren Blick auf die Mathematik als Sprache richten. Mit der Logik aus dem vorherigen Kapitel, soll dies dazu führen, dass ihr die Angst vor kompliziert anmutenden Formeln wie
\begin{equation*}
\forall \epsilon > 0 \ \exists \delta > 0 : \forall x \in \mathbb{D} \text{ mit } \left|x - x_0 \right| < \delta \text{ gilt: } \left| f(x) - f(x_0) \right| < \epsilon 
\end{equation*}
verliert.\\
Was auf den ersten Blick schwierig aussieht wird einfach(er), wenn man es wie einen normalen Satz lesen kann:
\textit{
Für alle positiven Zahlen Epsilon gibt es (mindestens) eine positive Zahl Delta, so dass für alle Zahlen x in D mit einem kleineren Abstand zu $x_0$ als Delta gilt, dass die Funktionswerte von f an der Stelle x weniger als Epsilon von den Funktionswerten von f an der Stelle $x_0$ entfernt sind.
}
Dieses Kapitel hilft, Formeln wie die obige in lesbare Sätze zu übersetzen, die dann (im nächsten Schritt) diskutiert und verstanden werden können.

\subsection{Die Mengenlehre}
Eine Menge ist eine \textbf{ungeordnete} Gruppe von \textbf{verschiedenen} Objekten. Ein Objekt einer Menge $M$ nennen wir \textit{Element} von $M$. Falls das Element $x$ \textbf{ein Element von $M$} ist so schreiben wir 
\begin{equation*}
x \in M.
\end{equation*}
Falls $x$ \textbf{nicht Element von $M$} ist, schreiben wir
\begin{equation*}
x \notin M.
\end{equation*}

\begin{warning}
	$(x \in M \lor x \notin M)$ ist eine Tautologie, denn folgendes ist immer wahr: Ein Element $x$ ist Element einer Menge $M$ oder es ist nicht Element von $M$.
\end{warning}

\subsubsection{Beispiele}
Mengen können in den unterschiedlichsten Schreibweisen auftreten:
\begin{itemize}
\item $M := \left\{a, b, c\right\}$ ist die Menge mit den Elementen $a,b,c$, also z.B. $b \in  \left\{a, b, c\right\}$ bzw. $b \in M$
\item $M := \left\{1, 2, 3, \ldots \right\}$ ist die unendliche Menge die alle ganzen positiven Zahlen $> 0$ enthält.
\item $X:= \left\{x \ | \ x < 8 \ \land \ x \text{ ist eine ganze positive Zahl} \right\}$,\\$M := \left\{x \in X \ | \ x \text{ ist eine Primzahl} \right\} = \left\{2, 3, 5, 7\right\}$.
\item $\mathbb{N}, \mathbb{Z}, \mathbb{R}, \ldots$ sind Symbole für Mengen die wir häufig verwenden.
\end{itemize}

\subsubsection{Mengenoperationen}
Mengen sind eng mit der Aussagenlogik verbunden. Ähnlich wie zwei Aussagen können wir auch zwei Mengen verknüpfen. Seien $A$ und $B$ zwei Mengen.
\begin{equation*}
A \cap B = \left\{ x \ | \ x \in A \ \land \ x \in B \right\}
\end{equation*}
Wir sehen hier, dass das $\cap$ in der Mengenlehre dem logischen $\land$ in der Aussagenlogik entspricht.
\begin{itemize}
\item Schnitt, sprich \textbf{A geschnitten B}: $A \cap B =  \left\{x \ | \ x \in A \ \land \ x \in B \right\}$
\item Vereinigung, sprich \textbf{A vereinigt mit B}: $A \cup B =  \left\{x \ | \ x \in A \ \lor \ x \in B \right\}$
\item Differenz, sprich \textbf{A ohne B}: $A \setminus B =  \left\{x \ | \ x \in A \ \land \ x \notin B \right\}$
\item Komplement, sprich \textbf{nicht A}: ($A \subseteq U$): $A^C = \bar{A} = \left\{x \in U \ | \ x \notin A \right\}$.
\begin{warning}
	Das Komplement macht nur Sinn, wenn wir ein sog. Universum $U$ haben, das A \textit{einschließt}.\\
	Für U gilt dann: $U := \bar{A} \cup A$
\end{warning}
\end{itemize}
Durch die Analogie zur Aussagenlogik gelten die gleichen \textbf{Rechenoperationen}, darum verzichten wir auf eine Auflistung. Ein Beispiel wären die \textbf{De Morgen-Regeln}:
\begin{gather*}
(A \cap B)^C = A^C \cup B^C \\
(A \cup B)^C = A^C \cap B^C 
\end{gather*} 

\subsubsection{Mengenrelationen}
Eine Relation stellt zwei mathematische Objekte in eine Beziehung/Relation. So etwa die $<$-Relation (sprich: kleiner-Relation).\\
$4 < 5$ besagt, dass $4$ in der $<$-Relation zu $5$ steht, also dass $4$ eben kleiner als $5$ ist. Wir benötigen aber auch Relationen für Mengen um diese z.B. zu vergleichen. Wir wollen Aussagen, dass alle Elemente von $A$ auch in $B$ enthalten sind oder andersherum.
\begin{itemize}
\item Teilmenge: $(A \subseteq B) \iff (x \in A \Rightarrow x \in B)$ quasi ($\leq$)
\item Supermenge: $A \supseteq B  \iff (x \in A \Leftarrow x \in B)$ quasi ($\geq$)
\item Echte Teilmenge: $(A \subset B) \iff (x \in A \Rightarrow x \in B \land A \neq B)$ quasi ($<$)
\item Echte Supermenge: $(\supset B) \iff (x \in A \Leftarrow x \in B \land A \neq B)$ quasi ($>$)
\end{itemize}

\begin{warning}
	Der Unterschied zwischen einer \textit{Teilmenge} und einer \textit{echten Teilmenge} ist, dass die \textit{echte Teilmenge} auf jeden Fall weniger Elemente als die zugehörige Supermenge besitzt.
\end{warning}

\subsubsection{Die leere Menge}
Die leere Menge ist jene, welche keinerlei Elemente enthält.
Für diese besondere Menge gibt es zwei Schreibweisen, welche sich durchgesetzt haben:
\begin{itemize}
	\item $\emptyset$
	\item $\{\}$
\end{itemize}
\begin{warning}
	Egal welche Menge betrachtet wird, die leere Menge ist stets eine Teilmenge von ihr.
\end{warning}