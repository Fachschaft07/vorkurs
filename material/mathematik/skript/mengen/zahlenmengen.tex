\subsection{Zahlenmengen}
Dieses Kapitel gibt einen kurzen überblick über die in der Mathematik üblichen Zahlenmengen.

\subsubsection{Natürliche Zahlen}
Die natürlichen Zahlen $\mathbb{N}$ umfassen alle ganzen Zahlen von 1 aufwärts.\\
$\mathbb{N} := \{1, 2, 3, ...\}$
\begin{warning}
	Ob die Zahl Null zu den natürlichen Zahlen gehört ist eine Streitfrage unter den Mathematikern. Um Klarheit zu verschaffen, wird in diesem Fall oft $\mathbb{N}_0$ ($\mathbb{N} mit 0$) verwendet. Dies ist jedoch nicht garantiert. Ihr solltet im Zweifelsfall euren Professor fragen, wie er dies handhabt.
\end{warning}

\subsubsection{Ganze Zahlen}
$\mathbb{Z} := \left\{\ldots, -3, -2, -1, 0, 1, 2, 3, \ldots \right\}$, diese Zahlen erweitern die natürlichen Zahlen um negative ganze Zahlen (und Null). Mit ihnen ist es möglich, uneingeschränkt zu subtrahieren.
\paragraph{Beispiel: } $3 - 4 = -1$

\subsubsection{Rationale Zahlen}
$\mathbb{Q} := \left\{ q \  : \ q = \frac{x}{y}, \ x,y \in \mathbb{Z} \  \land \ y \neq 0 \right\} \supset \mathbb{N}$, mit der Erweiterung auf die rationalen Zahlen sind alle vier Grundrechenarten inklusive der Division möglich.

\paragraph{Beispiele: } $\frac{1}{3}$,$-\frac{7}{13}$, $1 = \frac{1}{1}$, $-8 =\frac{-8}{1}$

\subsubsection{Reelle Zahlen}
$\mathbb{R} = \mathbb{Q} \cup \mathbb{I}$, wobei wir mit $\mathbb{I}$ die irrationalen Zahlen bezeichnen. Die irrationalen Zahlen bilden \textbf{unendliche}, \textbf{nicht periodische} und demzufolge nicht als Bruch darstellbare Zahlen.

\paragraph{Beispiele: } $\sqrt{2}$, $\sqrt[3]{17}$, $\pi$, $e$

%\subsection{Komplexe Zahlen}
%Symbol: $\mathbb{C}$
%\begin{flushleft}
%Die komplexen Zahlen sind der algebraische Abschluss der reellen Zahlen. Dies
%bedeutet, dass jedes Polynom eine Nullstelle hat. Es gibt folglich eine
%(nicht-reelle) Zahl $i \in \mathbb{C}$ mit $i^2 + 1 = 0$ bzw. $i^2=-1$, die
%imaginäre Einheit. Komplexe Zahlen bestehen aus einem reellen und einem imaginären Teil. Um komplexe Zahlen zu multiplizieren, benutzt man oft die Gaußsche Zahlenebene und die Polarform.
%\end{flushleft}
%\paragraph{Beispiele:} $5+3i \approx 5.83*e^{i*30.96}$, $4-5i$, $i^2=-1$

%\paragraph{Anmerkung:} Da die komplexe Zahlen in vielerlei Hinsicht verwendet werden können, werden diese hier nicht genauer formuliert. Auf spezielle Eigenschaften wird in den einzelnen Vorlesungen genauer eingegangen.

\subsubsection{Ordnung der Zahlenmengen} 
Mit Ausnahme der irrationalen Zahlen können die (hier behandelten) Zahlenmengen als Erweiterungen der jeweils vorhergehenden verstanden werden. $\mathbb{N}$ bildet dabei die Basis.
\begin{equation*}
 \mathbb{N} \subset \mathbb{Z} \subset \mathbb{Q} \subset \mathbb{R}
\end{equation*}

\subsubsection{Intervalle}
Ein Intervall $I$ ist eine Teilmenge bestehend aus allen Zahlen einer Zahlenmenge, welche zwischen einer unteren $a$ und einer oberen Grenze $b$ eingeschlossen sind.
\paragraph{Beispiel (informell):} \textit{Das Intervall I umfasst alle natürlichen Zahlen größer 1 und kleiner 6.}
% Diese Grenzen können auch plus/minus unendlich also $+ \infty$ oder $- \infty$  annehmen. Lieber rausnehmen - wenn das Intervall unendlich _enthält_, dann ist die Supermenge R vereinigt mit Unendlich... Gemeint ist wohl eher, dass die obere/untere Schranke (-)unendlich sein kann, nicht, dass das Intervall unendlich enthält.

\paragraph{Abgeschlossene Intervalle}
\begin{flushleft}
Um auszudrücken, dass die Variable $x$ einen Wert in einem gewissen Intervall $A$ mit der linken Grenze $a$ und der rechten Grenze $b$ hat, schreibt man es als \textbf{(ab)geschlossenes Intervall}; es schließt die beiden Werte \textbf{a} und \textbf{b} mit ein. 
\begin{equation*}
x \in \left[a;b \right]
\end{equation*}
\begin{warning}
	Oft werden $a$ und $b$ auch durch ein Komma ($x \in \left[a,b \right]$) getrennt, was sich aber bei Zahlen in deutscher Notation als ungeschickt herausstellen kann, da hier das Komma als Dezimaltrennzeichen verwendet wird.
\end{warning}
\end{flushleft}

\paragraph{Offene Intervalle}
\begin{flushleft}
Um beide Werte auszuschließen, schreibt man ein \textbf{offenes Intervall}
\begin{equation*}
x \in (a;b) \textsf{ bzw. } x \in \left]a;b \right[
\end{equation*}
\end{flushleft}

\begin{warning}
	Es gibt mehrere Notationen, Werte eines Intervalls auszugrenzen. Die am meisten gebrauchte Schreibweise ist die Notation mit runden Klammern.
\end{warning}

\paragraph{Halboffene Intervalle}
\begin{flushleft}
Zudem gibt es noch \textbf{halboffene Intervalle}, wie das \textbf{rechtsoffene Intervall}
\begin{equation*}
x \in \left[a;b \right) \textsf{ bzw. } x \in \left[a;b \right[
\end{equation*}
und das \textbf{linksoffene Intervall}
\begin{equation*}
x \in \left(a;b \right] \textsf{ bzw. } x \in \left] a;b \right]
\end{equation*}
\end{flushleft}

\begin{warning}
Ein Intervall ist eine Menge, deshalb können wir somit alle Mengenoperationen damit durchführen:
\begin{equation*}
\left[0.5;10\right] \cup \mathbb{N} \setminus \{20\}
\end{equation*}
beinhaltet zum Beispiel alle reellen Zahlen von $0.5$ bis $10$, sowie alle natürlichen Zahlen, außer $20$.
\end{warning}

\subsection{Aufgaben}
\begin{enumerate}
	\item Seien $A = \{1,3,4,5\}, B = \{4, 5, 6, 7\}$ gegeben, berechne $A \cap B, A \cup B, A \setminus B$
	\item Gilt das Distributiv Gesetz also $(A \cap B) \cap C) = A \cap (B \cap C)$ und $(A \cup B) \cup C) = A \cup (B \cup C)$? (inklusive Begründung).
	\item Begründe, gilt $A^C \setminus B^C == (A \setminus B)^C$?.
	\item Gilt $(A \subseteq B \ \land \ B \subseteq A) \iff (A = B)$?
	\item Seien $A, B$ zwei Mengen. Verwende die Mengenoperationen um die Menge zu konstruieren, die alle Elemente aus $A$ oder $B$ enthältt, die nicht zugleich in $A$ und in $B$ liegen. Um welche Operation handelt es sich?
	\item Man addiere eine Zahl aus $\mathbb N$ und eine aus $\mathbb R \setminus \mathbb N$. In welcher Menge liegt die neue Zahl? (ohne Rundung!)
	%\item Formulieren Sie eine Funktion (Wurzel), für deren Lösung sie i bräuchten.
	\item Zähle die fünf größten Elemente von $\mathbb Z \setminus \mathbb N_0$ auf!
	\item Definiere mit Hilfe der natürlichen Zahlen die Menge $\mathbb{Z}$ der ganzen Zahlen! $\mathbb Z$ darf innerhalb der Definition nicht verwendet werden!
	\item Schreibe folgende Mengen in Intervallschreibweise und vereinfache sie so gut es geht:
\begin{itemize}
	\item $\left[5;100\right] \cup \left(- \infty; 5\right)$
	\item $\left(3;5\right] \cup \left(0; 3\right)$
	\item $\mathbb{N} \setminus \left(9;15\right]$
	\item $\left(-1;8\right] \cap \left(9;15\right]$
\end{itemize}	

\end{enumerate}
