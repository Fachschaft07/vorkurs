Die Logik öffnete den Weg für die Informatik. Sie war immer eine Disziplin der Philosophie und hat eine sehr interessante Geschichte (von Aristoteles, Kant, Hegel, Boole, Russell, Gentzen, Skolem, Gödel, bis zu Turing und weiter) hinter sich gelassen. Die Aussagenlogik, als kleiner aber grundlegender Teil, ist fundamental um mathematische Ausdrücke zu verstehen und wird in allen Bereichen der Mathematik verwendet.
\begin{warning}
In der Informatik hat die Logik eine wichtige Bedeutung. In der technischen Informatik werden Recheneinheiten mit sogenannten logischen Gattern realisiert. Ein Gatter ist nichts anders als eine logische Verknüpfung. Zum Beispiel ist das Und-Gatter ein Gatter mit zwei Eingängen $x_{in}^1, x_{in}^2$ und einem Ausgang $x_{out}$.\\
Legt man nun Spannung an $x_{in}^1$ und \textit{gleichzeitig} $x_{in}^2$ an, so gibt das Gatter diese Spannung weiter, ansonsten nicht. Die Gatter bilden den Kern der Computerhardware wie z.B. Mikroprozessoren. Zudem ist die Aussagenlogik in jeder Programmiersprache eingebettet.
\end{warning}

\subsection{Aussagenlogik}
\begin{definition}[Aussage]
Eine Aussage ist ein als Satz formulierter Gedanke, dem man auf sinnvolle Weise einen Wahrheitswert zuordnen kann.
\end{definition}
%\begin{axiom}[Gesetz vom ausgeschlossenen Dritten]
Eine Aussage ist entweder \textbf{wahr (w)} oder \textbf{falsch (f)} (Gesetz vom ausgeschlossenen Dritten).
%\end{axiom}

\subsubsection*{Beispiele}
\begin{itemize}
\item $A_1 := $ Die Sonne strahlt wärme ab.
\item $A_2 := $ Bayern München wird diese Saison wieder deutscher Meister.
\item $A_3 := 1+1 = 2$
\item $A_4 := $ Wird in diesem Jahr Ostern gefeiert?
\end{itemize}
$A_1$ und $A_3$ sind offensichtlich wahre Aussagen. $A_2$ können wir derzeit nicht überprüfen, sie ist aber trotzdem wahr oder falsch und darum zulässig. $A_4$ ist keine Aussage, sondern eine Frage - die Antwort darauf kann wieder eine Aussage sein (z.B. \textit{In diesem Jahr wird Ostern gefeiert.}).

\subsubsection*{Schreibweisen}
Wir schreiben gewöhnlich $A$ ist \textbf{wahr} oder \textbf{falsch}, \textbf{w} oder \textbf{f} bzw. (für uns Informatiker) \textbf{1} oder \textbf{0}.

\subsubsection{Logische Verknüpfungen}
\begin{definition}[Negation]
Unter der Negation einer Aussage $A$ versteht man die Aussage $\neg A$ (Mathe) bzw. $\overline{A}$ (Informatik) (in Worten:\glqq nicht $A$\grqq ), die genau dann wahr ist, wenn $A$ selbst falsch ist.
\end{definition}
Es folgt eine sogenannte \textbf{Wahrheitstabelle} der Verknüpfung. In der ersten Spalte stehen die Werte, die $A$ annimmt, und in der rechten Spalte, Werte, die $\neg A$ dann besitzt.
\begin{center}
\begin{tabular}{c||c}
 $A$ & $\neg A$  \\ 
\hline
\cellcolor{ared}falsch  & \cellcolor{agreen}wahr   \\ 
\cellcolor{agreen}wahr  & \cellcolor{ared} falsch  \\ 
\hline
\end{tabular}
\end{center}

\paragraph*{Beispiel:}
\begin{itemize}
	\item Die Negation der Aussage \glqq 4 ist ungerade\grqq \ ist die Aussage \glqq 4 ist gerade\grqq , denn es gibt nur diese beiden Möglichkeiten.
	\item Aber die Negation der Aussage \glqq 4.5 ist ungerade\grqq \ ist nicht die Aussage \glqq 4.5 ist gerade\grqq, denn beide Aussagen sind falsch, ja sogar unsinnig. Die Negation der Aussage \glqq Diese Kuh ist schwarz\grqq \ ist nicht etwa die Aussage \glqq Diese Kuh ist weiß\grqq , denn es gibt ja noch andere Farben. Vielmehr müsste man sagen:\glqq Diese Kuh ist nicht schwarz\grqq. Das umgangssprachliche Gegengenteil ist meist etwas anderes als die logische Verneinung.
\end{itemize}

\begin{definition}[Konjunktion]
Unter der Konjunktion zweier Aussagen $A$ und $B$ versteht man die Aussage $A \land B$ (Mathe) bzw. $A \cdot B$ (Informatik) (in Worten:\glqq $A$ und $B$\grqq ), die genau dann wahr ist, wenn $A$ und $B$ gleichzeitig wahr sind.
\end{definition}
In den ersten Spalten dieser \textbf{Wahrheitstabelle} stehen die Werte der Variablen der Konjunktion, in der letzten Spalte der resultierende Wert.
\begin{center}
\begin{tabular}{c|c||c}
 $A$& $B$  &  $A \land B$  \\ 
\hline
\cellcolor{ared} falsch & \cellcolor{ared} falsch & \cellcolor{ared} falsch   \\ 
\cellcolor{ared} falsch & \cellcolor{agreen} wahr & \cellcolor{ared} falsch  \\ 
\cellcolor{agreen} wahr & \cellcolor{ared} falsch & \cellcolor{ared} falsch   \\ 
\cellcolor{agreen} wahr & \cellcolor{agreen} wahr & \cellcolor{agreen} wahr  \\ 
\hline
\end{tabular}
\end{center}

\paragraph*{Beispiel:}
\begin{itemize}
	\item \glqq 18 ist eine gerade Zahl und durch 3 teilbar\grqq , ist eine wahre Aussage im Rahmen des Axiomensystems für die Arithmetik, welche hier umgangssprachlich vorausgesetzt wurde. Eigentlich handelt es sich um die Aussage \glqq 18 ist eine gerade Zahl, und 18 ist durch 3 teilbar\grqq .
	\item \glqq 15 ist eine gerade Zahl und durch 3 teilbar\grqq, ist eine falsch, denn der erste Teil der Aussage ist falsch.
\end{itemize}

\begin{definition}[Disjunktion]
Unter der Disjunktion zweier Aussagen $A$ und $B$ versteht man die Aussage $A \lor B$ (Mathe) bzw. $A + B$ (Informatik) (in Worten:\glqq $A$ oder $B$\grqq ), die genau dann wahr ist, wenn wenigstens eine der Aussagen $A$ oder $B$ wahr ist.
\end{definition}
In den ersten Spalten dieser \textbf{Wahrheitstabelle} stehen die Werte der Variablen der Disjunktion, in der letzten Spalte der resultierende Wert.
\begin{center}
\begin{tabular}{c|c||c}
$A$& $B$  &  $A \lor B$  \\ 
 \cellcolor{ared}falsch & \cellcolor{ared} falsch & \cellcolor{ared}falsch   \\ 
 \cellcolor{ared}falsch & \cellcolor{agreen}wahr & \cellcolor{agreen}wahr  \\ 
 \cellcolor{agreen}wahr & \cellcolor{ared} falsch & \cellcolor{agreen}wahr   \\ 
\cellcolor{agreen}wahr & \cellcolor{agreen}wahr & \cellcolor{agreen}wahr  \\ 
\hline
\end{tabular}
\end{center}
\begin{warning}
	Umgangssprachlich meinen wir mit \glqq oder\grqq \ meist \glqq entweder $\ldots$ oder\grqq . Was aber nicht der logischen Disjunktion entspricht.
\end{warning}

\paragraph*{Beispiel:}
\begin{itemize}
	\item \glqq Ich werde Mathematik oder Informatik studieren\grqq , diese Aussage ist auch dann wahr, wenn ich mich dafür entscheide, beide Fächer zu studieren. Falsch wird sie aber z.B., wenn ich nur Biologie studiere.
	\item \glqq Ich kann nur Hü oder Hott sagen\grqq . Das ist natürlich falsch, denn ich kann ja beide Wörter vermeiden.
\end{itemize}

\begin{definition}[Implikation]
Unter der Implikation zweier Aussagen $A$ und $B$ versteht man die Aussage $A \Rightarrow B$ (in Worten:\glqq $A$ impliziert $B$\grqq oder \glqq aus $A$ folgt $B$\grqq ), versteht man die Zusammengesetzte Aussage $ (\neg A) \lor B$.
\end{definition}
In den ersten Spalten dieser \textbf{Wahrheitstabelle} stehen die Werte der Variablen der Implikation, in der letzten beiden Spalte der resultierenden Werte (in der letzten der finale Wert).
\begin{center}
\begin{tabular}{c|c||c||c}
$A$& $B$ &$\neg A$&  $\neg A \lor B$  \\ 
 \cellcolor{ared}falsch &  \cellcolor{ared}falsch & \cellcolor{agreen}wahr  & \cellcolor{agreen}wahr  \\ 
 \cellcolor{ared}falsch & \cellcolor{agreen}wahr & \cellcolor{agreen}wahr  &  \cellcolor{agreen}wahr  \\ 
\cellcolor{agreen}wahr &  \cellcolor{ared}falsch & \cellcolor{ared} falsch  &  \cellcolor{ared}falsch  \\ 
\cellcolor{agreen}wahr & \cellcolor{agreen}wahr & \cellcolor{ared} falsch  &  \cellcolor{agreen}wahr  \\ 
\hline
\end{tabular}
\end{center}

$A$ wird auch als Prämisse bezeichnet. Eigentlich sieht alles recht einfach aus. Nehmen wir die Aussagen: ($A$) \glqq Wenn es regnet folgt daraus, dass ($B$) die Stra{ß}e nass wird\grqq . Wenn es nun regnet und die Stra{ß}e nass wird ist die Aussage wahr. Doch was passiert wenn es nicht regnet?\\
Wenn es nicht regnet ist die Aussage immer wahr egal ob die Stra{ß}e nun nass oder trocken ist.
Wir können die Aussage auch wie folgt formulieren:
\begin{center}
\glqq Ist die Stra{ß}e nicht nass, so folgt daraus, dass es nicht regnet.\grqq\\ (Kontropositionsgesetz)
\end{center}
\begin{equation*}
((\neg B) \Rightarrow (\neg A)) \iff (A \Rightarrow B)
\end{equation*}

\begin{definition}["Aquivalenz]
Unter der "Aquivalenz zweier Aussagen $A$ und $B$ versteht man die Aussage $A \iff B$ (in Worten:\glqq $A$ gilt genau dann wenn $B$ gilt\grqq ). Diese ist genau dann wahr wenn $A \Rightarrow B \land B \Rightarrow A$ wahr ist. 
\end{definition}
In den ersten Spalten dieser \textbf{Wahrheitstabelle} stehen die Werte der Variablen der "Aquivalenz, in der letzten Spalte der resultierende Wert.
\begin{center}
\begin{tabular}{|c|c||c|}
 $A$ & $B$ & $A \iff B$  \\ 
\cellcolor{ared}falsch & \cellcolor{ared}falsch & \cellcolor{agreen}wahr    \\ 
\cellcolor{ared}falsch & \cellcolor{agreen}wahr & \cellcolor{ared}falsch    \\ 
\cellcolor{agreen}wahr & \cellcolor{ared}falsch & \cellcolor{ared}falsch    \\ 
\cellcolor{agreen}wahr & \cellcolor{agreen}wahr & \cellcolor{agreen}wahr    \\ 
\hline
\end{tabular}
\end{center}
Ein Beispiel wäre $x + 5 = 7 \iff x + 8 = 10$.


\begin{definition}[Exklusives Oder]
Unter dem exklusives Oder zweier Aussagen $A$ und $B$ versteht man die Aussage $A \text{ XOR } B$ (in Worten:\glqq entweder $A$ oder $B$\grqq ). Diese ist genau dann wahr wenn $(A \land \neg B) \lor (\neg A \land B)$ wahr ist. 
\end{definition}
In den ersten Spalten dieser \textbf{Wahrheitstabelle} stehen die Werte der Variablen des exklusiven Oder, in der letzten Spalte der resultierende Wert.
\begin{center}
\begin{tabular}{|c|c||c|}
$A$ & $B$ & $A \textsf{ XOR } B$  \\ 
\cellcolor{ared}falsch &  \cellcolor{ared}falsch &  \cellcolor{ared}falsch    \\ 
\cellcolor{ared}falsch & \cellcolor{agreen}wahr & \cellcolor{agreen}wahr    \\ 
\cellcolor{agreen}wahr & \cellcolor{ared}falsch & \cellcolor{agreen}wahr    \\ 
\cellcolor{agreen}wahr & \cellcolor{agreen}wahr & \cellcolor{ared}falsch    \\ 
\hline
\end{tabular}
\end{center}
\begin{warning}
	Das exklusive Oder ist in die meisten Programmiersprachen, als Standardfunktion, eingebaut ist.
\end{warning}

\subsubsection{Grundlegende Rechenregeln (Bonus)}
Wir verwenden das Symbol $\equiv$ anstatt $\iff$ um anzudeuten, dass wir zwei Ausdrücke vergleichen möchten und diese sollen durch das $\equiv$-Symbol besser abgetrennt werden. Die Symbole können als identisch angesehen werden. Wir möchten mit dem $\equiv$-Symbol verdeutlichen, dass die eine durch die andere Formel ersetzt/vereinfacht werden kann, da sie äquivalent sind.
\begin{itemize}
\item Neutrales Element (bezgl. oder): $A \ \lor $ falsch $\equiv A$
\item Neutrales Element (bezgl. und): $A \ \land $ wahr $\equiv A$
\item Absorbierendes Element (bezgl. oder): $A \ \lor $ wahr $\equiv$ wahr
\item Absorbierendes Element (bezgl. und): $A \ \land $ falsch $\equiv$ falsch
\item Kommutativgesetz 1: $A \lor B \equiv B \lor A$
\item Kommutativgesetz 2: $A \land B \equiv B \land A$
\item Assoziativgesetz 1: $(A \lor B) \lor C \equiv A \lor (B \lor C)$
\item Assoziativgesetz 2: $A \land B) \land C \equiv A \land (B \land C)$
\item Distributivgesetz 1: $(A \land B) \lor C \equiv A \lor C \land  B \lor C$
\item Distributivgesetz 2: $(A \lor B) \land C \equiv A \land C \lor B \land C$
\item Idempotenzgesetz 1: $A \lor A \equiv A$
\item Idempotenzgesetz 2: $A \land A \equiv A$
\item Negation 1: $\neg \neg A \equiv A$
\item Tautologie: $\neg A \lor A \equiv $ wahr
\item Widerspruch: $\neg A \land A \equiv$ falsch
\item Absorbtionsgesetz 1: $A \lor (A \land B) \equiv A$
\item Absorbtionsgesetz 2: $A \land (A \lor B) \equiv A$
\item De-Morgan-Gesetzt 1: $\neg (A \lor B) \equiv \neg A \land \neg B$ (sehr nützlich)
\item De-Morgan-Gesetzt 2: $\neg (A \land B) \equiv \neg A \lor \neg B$ (sehr nützlich)
\end{itemize}
Man bezeichnet wahr auch als das \textbf{neutrale} Element der \textbf{Konjunktion} (ähnlich der 1 in der Multiplikation) und falsch als das \textbf{neutrale} Element der \textbf{Disjunktion} (ähnlich der 0 in der Addition). Dies Erklärt auch die Informatikschreibweise denn verwenden wir anstatt wahr die 1 und anstatt falsch die 0, so können wir mit einer Aussageformel rechen: $A \land B \equiv A \cdot B$, $A \lor B \equiv A + B \equiv \min(A + B, 1)$, das aber nur am Rande.

\subsubsection{Tautologie}
Als Tautologie bezeichnet man eine Aussage die immer wahr ist. So zum Beispiel das Gesetz vom ausgeschlossenen Dritten

\paragraph*{Beispiel:}
\begin{equation*}
A \lor (\neg A)
\end{equation*}

\subsubsection{Widerspruch}
Als Widerspruch bezeichnet man eine Aussage die immer falsch ist. 

\paragraph*{Beispiel}
\begin{equation*}
(\neg A) \land A
\end{equation*}

\subsubsection{Aufgaben}
Werten folgende Ausdrücke aus:
\begin{enumerate}
	\item wahr $\land$ falsch $\lor$ wahr
	\item wahr $XOR$ (wahr $\lor$ falsch)
	\item wahr $XOR$ (wahr $\land$ falsch)
	\item falsch $XOR$ falsch $\lor$ wahr $\Rightarrow$ wahr
	\item (falsch $XOR$ falsch $\land$ wahr) $\Rightarrow$ falsch
	\item falsch $\iff$ wahr
	\item (wahr $\Rightarrow$ wahr) $\iff$ (falsch $\Rightarrow$ falsch)
\end{enumerate}
Vereinfache folgende Ausdrücke:
\begin{enumerate}
	\item $A \lor (B \land A) \lor (C \land A) \lor (D \land B)$
	\item $A \lor \neg A \land (C \land \neg C)$
	\item $(A \Rightarrow B) \land (\neg B \Rightarrow \neg A)$
	\item $A \land (A \lor B)$
\end{enumerate}
Führe folgende Negationen aus:
\begin{enumerate}
	\item $\neg (\neg C \lor \neg D)$
	\item $\neg ((A \lor B) \land (C \lor D))$
\end{enumerate}
Zeige, dass die folgenden Aussageverknüpfungen Tautologien sind:
\begin{enumerate}
	\item $(A \land (A \Rightarrow B)) \Rightarrow B$ (Abtrennungsregel)
	\item $((A \Rightarrow B) \land (B \Rightarrow C)) \Rightarrow (A \Rightarrow C)$ (Syllogismus-Regel)
	\item $(A \Rightarrow B) \iff (\neg B \Rightarrow \neg A)$ (Kontrapositionsgesetz)
\end{enumerate}

%\subsubsection*{Syntax der Aussagenlogik}
Eine atomare Formel hat die Form $A_i$, wobei $i = 1,2,3,\ldots$. Formeln werden durch folgenden induktiven Prozesse definiert:
\begin{enumerate}
\item Alle atomaren Formeln sind Formeln.
\item Für alle Formeln $F$ und $G$ sind $(F \land G)$ und $(F \lor G)$ Formeln.
\item Für jede Formel $F$ ist $\neg F$ eine Formel.
\end{enumerate}

%\subsubsection{Semantik der Aussagenlogik}
%Die Elemente der Menge $\{\text{wahr}, \text{falsch}\}$ heißen Wahrheitswerte. Eine Belegung ist eine Funktion $\mathcal{A} : D \to \{\text{wahr}, \text{falsch}\}$, wobei $D$ eine Teilmenge der Formeln ist.

%\subsubsection*{Theorien}
%Wenn wir beweisen wollen, dass eine Aussage $B$ wahr ist, und wenn wir schon wissen, dass Aussage $A$ wahr ist und dass $B$ von $A$ impliziert wird, wie gehen wir dann vor? Es sei $\mathcal{A}$ das System der Axiome unserer Theorie. Das sind einfach Aussagen, deren Wahrheit wir einfach festsetzen. Zu ihrer Formulierung benutzen wir primitive Terme und logische Verknüpfungen. Mit Hilfe dieser primitiven Terme und logischen Verknüpfungen lassen sich weitere Aussagen formulieren, die entweder wahr oder falsch sind. Mit $\mathcal{W}$ sei die Gesamtheit derjenigen Aussagen bezeichnet, die unter Voraussetzung von $\mathcal{A}$ wahr sind. Theoretisch lassen sich diese wahren Aussagen mit Hilfe von Wahrheitstafeln bestimmen. Praktisch wird das meist nicht so einfach gehen, man müsste mit unendlich gro{ß}en Tafeln arbeiten. Deshalb führen wir ein Regelsystem zur Bildung von Ketten wahrer Aussagen $A_1, A_2, \ldots$ (sogenannter Beweise) ein, und wir sehen die Wahrheit einer Aussage $A$ als bewiesen an, wenn sie am Ende einer solchen Kette steht. Wir sagen dann, $A$ ist aus $\mathcal{A}$ ableitbar und schreiben:
%\begin{equation*}
% \mathcal{A} \vdash A
%\end{equation*}
%Eine Beweiskette $A_1, A_2, \ldots, A_n$ wollen wir zulässig nennen, wenn jede Aussage $A$, die in der Kette vorkommt, eine der folgenden Bedingungen erfüllt:
%\begin{enumerate}
%\item $A$ ist eines der Axiome
%\item $A$ ist eine Tautologie ($\vdash A$)
%\item Es gibt eine Kette vor $A$ eine Aussage $B$ und die Aussage $B \Rightarrow A$. Diese Beweis-Schema trägt auch die Bezeichnung modus ponens und kann so beschrieben werden:
%Wenn $\mathcal{A} \vdash B$ und $\mathcal{A} \vdash (B \Rightarrow A$ dann $\mathcal{A} \vdash A$ (direkter Beweis)
%\item $A$ entsteht aus einer vorher abgeleiteten zusammengesetzten Aussage $B$, indem man eine in $B$ auftauchende Teilaussage durch eine äquivalente ersetzt. Dieses Ersetzungs-Prinzip verleiht den Tautologien eine besondere Bedeutung.
%\item $A$ ist eine Definition, führt also nur eine andere Bezeichnung für etwas bekanntes ein, wir verwenden hierfür $:=$ oder $:\iff$.
%\end{enumerate}

%\subsubsection*{Konstruktivismus}
%Es gibt logische System, neben der klassischen Logik, in denen das Gesetzt vom ausgeschlossenen Dritten \textbf{nicht} gilt, solche Systeme vermeiden damit den Beweis durch Widerspruch. Was erst einmal irrsinnig klingt macht Sinn wenn man bedenkt, dass z.B. durch das Gesetzt die Existenz von Objekten beweisen können ohne diese 'anzufassen' indem wir von der Nicht-Existenz ausgehen und damit zu einem Widerspruch gelangen (Stichwort: Konstruktivismus). Der Konstruktivismus ist für Informatiker von großer Bedeutung. Die Beweise in der Informatik (sehen sie später in der theoretischen Informatik) sind fast immer konstruktiv. Solche Beweise liefern als Nebenprodukt oft einen konkreten Lösungsweg bzw. einen Algorithmus mit dem man etwas anfangen kann. Andere Beweise mögen einem eine Aussage beantworten, sonst haben sie aber keinerlei Mehrwert.
