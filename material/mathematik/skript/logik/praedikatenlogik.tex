\subsection{Pr"adikatenlogik (Bonus)}
Dies ist eine sehr kurz gehaltene Einf"uhrung in die Pr"adikatenlogik. Kurz gesagt ist die \textbf{Pr"adikatenlogik} \textbf{$\approx$} \textbf{Aussagenlogik mit Quantoren}. Wir m"ochten einem Objekt eine Eigenschaft/\textbf{Pr"adikat} zuweisen z.B. \glqq$7$ ist eine Primzahl\grqq . In diesem Fall ist das Pr"adikat \textit{Primzahl} und $7$ das \textit{Objekt}. Wir schreiben 
\begin{equation*}
P(x)
\end{equation*}
f"ur $x$ erf"ullt Pr"adikat $P$ (z.B. \glqq ist Primzahl\grqq ). Pr"adikate k"onnen eine unterschiedliche Stelligkeit besitzen ($P(x_1, x_2, \ldots, x_n)$, $n$-stelliges Pr"adikat). Um Elementen/Objekten eine Eigenschaft zuzuweisen ben"otigen wir noch eine Menge $M$ an Objekten. Wir schreiben dann:
\begin{equation*}
\underbrace{\exists}_{\text{es existiert ein}} \underbrace{x}_{\text{x}} \underbrace{\in}_{\text{in}} \underbrace{M}_{\text{M}} \underbrace{:}_{\text{sodass}} \underbrace{P(x)}_{\text{x die Eigenschaft P besitzt.}}
\end{equation*}
oder
\begin{equation*}
\underbrace{\forall}_{\text{f"ur alle}} \underbrace{x}_{\text{x}} \underbrace{\in}_{\text{in}} \underbrace{M}_{\text{M}} \underbrace{:}_{\text{gilt dass}} \underbrace{P(x)}_{\text{x die Eigenschaft P besitzt.}}
\end{equation*}

\subsubsection*{Beispiel}
Sei $T$ die Menge aller Tomaten und das Pr"adikat $R(x):= \text{\glqq}x \text{ ist rot\grqq}$ gegeben, so kommen wir von der klassischen Aussage $A := \text{\glqq Tomaten sind rot\grqq}$, nun zu:
\begin{itemize}
\item \glqq\textbf{alle} Tomaten sind rot\grqq : $\forall x \in T : R(x)$
\item \glqq\textbf{keine} Tomate ist rot\grqq : $\forall x \in T : \neg R(x)$ (bzw. $\neg (\exists x \in T : R(x))$)
\item \glqq\textbf{es exisitiert mindestens eine} Tomate die rot ist\grqq : $\exists x \in T : R(x)$ (bzw. $\neg (\forall x \in T : \neg R(x))$)
\item \glqq\textbf{nicht alle} Tomaten sind rot\grqq : $\exists x \in T : \neg R(x)$ (bzw. $\neg (\forall x \in T : R(x))$)
\end{itemize}
Man bedenke:
\begin{itemize}
\item $ \neg (\forall x \exists y P(x, y)) \equiv \exists x \forall y \neg P(x, y)$
\item $ \neg (\exists x \forall y P(x, y)) \equiv \forall x \exists y \neg P(x, y)$
\end{itemize}
Die Verneinung der Aussage \textbf{\glqq alle Tomaten sind rot\grqq} ($\neg (\forall x \in T : R(x))$) ist eben nicht \glqq keine Tomate ist rot\grqq \ $\forall x \in T : \neg R(x)$ sondern \textbf{\glqq es gibt mindestens eine Tomate dir nicht rot ist\grqq} $ \exists x \in T : \neg R(x)$!

\subsubsection{Aufgaben}
F"uhren Sie folgende Negationen aus:
\begin{itemize}
	\item $\neg A, \text{ mit } A:= $ Kein Mensch kann Fu{\ss}ballspielen.
	\item $\neg A, \text{ mit } A := $ Alle Rosen sind rot.
	\item $\neg A, \text{ mit } A := $ Alle Tomaten schmecken klasse und sind rot.
	\item $\neg A, \text{ mit } A := $ Wenn jemand im Lotto gewinne folgt daraus, dass diese Person viel Geld hat-
\end{itemize}
Sei nun $M$ die Menge aller Menschen, $R$ die Menge aller Rosen, $T$ die Menge aller Tomaten. Definieren Sie geeignete Pr"adikate und formulieren Sie m"oglichst kurze Aussagen der Pr"adikatenlogik, welche den obigen Aussagen entsprechen. F"uhren Sie anschlie"send die gleiche Negation durch.