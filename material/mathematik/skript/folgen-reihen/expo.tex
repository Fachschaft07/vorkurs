\subsubsection{Rechnen mit Exponenten}
Wie ihr vielleicht schon bemerkt habt, gibt es bei den \textcolor{red}{geometrischen Folgen} eine neue mathematische Schreibweise ($q^n$).\\
Die Konstante q wird hierbei als \textcolor{red}{Basis} bezeichnet, w"ahrend n \textcolor{red}{Exponent} genannt wird.\\
Hierbei gilt:\\
\begin{itemize}
\item $q^0 = 1$
\item $q^1 = q$
\item $q^2 = q*q$
\item $q^3 = q*q*q$
\item $q^4 = q*q*q*q$
\item ...
\end{itemize}

\begin{center}
\textbf{Rechengesetze f"ur Exponenten}\\
\end{center}
\begin{tabular}{|>{\centering\arraybackslash}p{6.5 cm}|>{\centering\arraybackslash}p{6.5 cm}|}
\hline
\textsc{Urspr"unglich}&\textsc{Vereinfacht}\\
\hline
$((b)^n)^m$&$b^{n*m}$\\
\hline
$b^n * b^m$&$b^{n+m}$\\
\hline
$\frac{b^n}{b^m}$&$b^{n-m}$\\
\hline
\end{tabular}

\begin{center}
\textbf{Rechengesetze f"ur Wurzeln}
\end{center}
\begin{tabular}{|>{\centering\arraybackslash}p{6.5 cm}|>{\centering\arraybackslash}p{6.5 cm}|}
\hline
\textsc{Urspr"unglich}&\textsc{Vereinfacht}\\
\hline
$\sqrt[m]{a}$&$a^{\frac{1}{m}}$\\
\hline
$\sqrt[m]{a}*\sqrt[m]{b}$&$\sqrt[m]{a*b}$\\
\hline
$\frac{\sqrt[m]{a}}{\sqrt[m]{b}}$&$\sqrt[m]{\frac{a}{b}}$\\
\hline
$\sqrt[m]{b^n}$&$b^{\frac{n}{m}}$\\
\hline
\end{tabular}