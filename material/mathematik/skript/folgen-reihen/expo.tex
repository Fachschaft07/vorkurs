\subsubsection{Rechnen mit Exponenten}
Wie ihr vielleicht schon bemerkt habt, gibt es bei den \textcolor{red}{geometrischen Folgen} eine neue mathematische Schreibweise ($q^n$). Die Konstante $q$ wird hierbei als \textcolor{red}{Basis} bezeichnet, während $n$ \textcolor{red}{Exponent} genannt wird. Hierbei gilt:
\begin{itemize}
\item $q^0 = 1$
\item $q^1 = q$
\item $q^2 = q \cdot q$
\item $q^3 = q \cdot q \cdot q$
\item $q^4 = q \cdot q \cdot q \cdot q$
\item $\ldots$
\end{itemize}

\paragraph{Rechengesetze für Exponenten}
\begin{enumerate}
\item $((b)^n)^m = b^{n \cdot m}$
\item $b^n \cdot b^m = b^{n+m}$
\item $\frac{b^n}{b^m} = b^{n-m}$
\end{enumerate}

\paragraph{Rechengesetze für Wurzeln}
\begin{enumerate}
\item $\sqrt[m]{a} = a^{\frac{1}{m}}$
\item $\sqrt[m]{a} \cdot \sqrt[m]{b} = \sqrt[m]{a \cdot b}$
\item $\frac{\sqrt[m]{a}}{\sqrt[m]{b}} = \sqrt[m]{\frac{a}{b}}$
\item $\sqrt[m]{b^n} = b^{\frac{n}{m}}$
\end{enumerate}