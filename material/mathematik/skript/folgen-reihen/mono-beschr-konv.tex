\subsection{Monotonie, Beschränktheit und Konvergenz}
\subsubsection{Monotonie}
Eine Folge ist \textcolor{red}{monoton fallend}, wenn kein Folgeglied größer ist als dessen Vorgänger:
\begin{equation*}
\forall n \in \mathbb{N} : a_n \leq a_{n-1}
\end{equation*}
Ist eine Folge \textcolor{red}{monoton steigend}, so ist jedes Folgeglied entweder größer oder gleich dessen Vorgänger:
\begin{equation*}
\forall n\in{\mathbb{N}}: a_n \geq a_{n-1}
\end{equation*}

\paragraph{Strenge Monotonie} 
Der Unterschied zwischen Monotonie und strenger Monotonie ist einfach. Bei Monotonie d"urfen die Folgeglieder gleich deren Vorgänger sein, bei strenger Monotonie ist dies jedoch nicht der Fall:
\begin{itemize}
\item \textcolor{red}{streng monoton steigend} $\Leftrightarrow \forall n\in\mathbb{N}: a_n > a_{n-1}$
\item \textcolor{red}{streng monoton fallend} $\Leftrightarrow \forall n\in\mathbb{N}: a_n < a_{n-1}$
\end{itemize}

\subsubsection{Beschränktheit}
Ist eine Folge nach unten beschränkt, so gilt:
\begin{equation*}
\exists s \in \mathbb{R} \text{ so dass } \forall n \in \mathbb{N}: s \leq a_n.
\end{equation*}
Hierbei wird jeder Wert, welcher kleiner gleich des kleinsten Folgegliedes der Folge ist, als \textcolor{red}{untere Schranke} bezeichnet. Die größte untere Schranke wird als \textcolor{red}{Infimum} bezeichnet. Eine Folge ist dann nach oben beschränkt, wenn es mindestens einen Wert s aus $\mathbb{N}$ gibt, welcher größer oder gleich des größten Folgegliedes ist. Diese Werte bezeichnet man als \textcolor{red}{obere Schranke}:
\begin{equation*}
\exists s \in \mathbb{R} \text{ so dass } \forall n \in \mathbb{N} : s\geq a_n 
\end{equation*}
Die kleinste obere Schranke heißt \textcolor{red}{Supremum}.

\subsubsection{Konvergenz}
Nähert sich eine Folge stetig einem bestimmten "'\textcolor{red}{Grenzwert}"' (oder "'\textcolor{red}{Limes}"') $a$ beliebig nahe an, so sagt man, sie konvergiert gegen $a$:
\begin{equation*}
\forall \epsilon > 0 \ \exists n_0  \in \mathbb{N} \text{ so dass } \forall n \in \mathbb{N} \text{ mit } n \geq n_0 : |a_n - a|< \epsilon
\end{equation*}
Epsilon $\epsilon$ ist hierbei eine beliebig kleine Zahl, welche den Abstand/die \textit{Umgebung} zwischen dem Wert von $a_n$ und dem Grenzwert $a$ beschreibt. Jetzt mal langsam. Was besagt der mathematische Ausdruck? Sei $a$ unser Grenzwert dann gibt es f"ur jedes positive Epsilon, egal wie klein es auch ist (nur null darf es nicht sein), eine nat"urliche Zahl $n_0$, sodass der Abstand zwischen $a$ alle Folgeglieder, die nach oder gleich dem $n_0$-ten Glied kommen, kleiner ist als Epsilon. Das heißt, wenn eine Folge gegen $a$ konvergiert kann ich euch ein $\epsilon > 0$ geben, zum Beispiel $0.00000000001$, und ihr könnt mir ein $n_0$ sagen, sodass $|a_n - a| < \epsilon$ und zwar f"ur alle $n \geq n_0$. Jede nicht konvergente Folge wird als \textcolor{red}{divergent} bezeichnet.

\paragraph{Beispiel}
Sei unsere Folge $a_n = \frac{1}{n}$. Wir nehmen an, dass die Folge gegen $0$ konvergiert. Ich gebe euch nun ein $\epsilon$, nun soll gelten 
\begin{equation*}
|a_n - 0| < \epsilon \iff \left|\frac{1}{n}\right| < \epsilon \stackrel{\text{da } n > 0}{\iff} \frac{1}{n} < \epsilon
\end{equation*}
lösen wir doch einfach nach $n$ auf:
\begin{equation*}
\frac{1}{n} < \epsilon \iff n > \frac{1}{\epsilon}
\end{equation*}
Nun könnt ihr mir immer ein $n$ nennen. Das $n_0$ ist hierbei mit irgendeinem $n$, was die Bedingung erf"ullt identisch und alle $n \geq n_0$ erf"ullen die Bedingung ebenso.

\subsubsection{Zusammenhänge}
\begin{itemize}
\item Jede konvergente Folge ist auch beschränkt, allerdings ist \textbf{nicht} jede beschränkte Folge auch konvergent. \textcolor{red}{Konvergenz $\Rightarrow$ Beschränktheit}
\item Jede beschränkte, monotone Folge ist konvergent. \textcolor{red}{Beschränktheit und Monotonie $\Rightarrow$ Konvergenz}
\end{itemize}