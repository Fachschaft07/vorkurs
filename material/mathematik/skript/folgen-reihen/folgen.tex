\subsection{Folgen}
Eine Folge ist eine Auflistung $a_1,a_2,a_3, \ldots$ von endlich bzw. unendlich vielen durchnummerierten Objekten. Formal ist eine Folge eine \textcolor{red}{Abbildung} $a : \mathbb{N} \to X$ von den \textcolor{red}{natürlichen Zahlen} $\mathbb{N}$ in eine Menge $X$ wobei $X$ meist gleich oder eine Teilmenge von $\mathbb{R}$ ist. Wir schreiben allerdings anstatt $a(i)$, $a_i$. Besonders interessant sind unendliche Folgen. Eine Folge wird entweder direkt, als \textit{Funktion} angegeben oder über ein \textit{rekursives Bildungsgesetzt}.

\paragraph{Beispiele:}
\begin{itemize}
\item $a_n = \frac{1}{n}$ (wir könnten hier auch schreiben $a(n) = \frac{1}{n}$ oder $n \mapsto \frac{1}{n}$)
\item $a_n = \frac{1}{n^{2}}$
\item $a_n = n^{2}$
\item $a_n = \underbrace{ggggg \cdots ggggg}_{n-\text{mal}}$ (hier wäre $a: \mathbb{N} \to \{g, gg, ggg, gggg, \ldots\}$)
\item 1,2,4,8,16,32,...
\item \begin{tabbing} 
\hspace{3.5 cm}\=\hspace{5 cm}\kill 
$a_{n+1} = 2a_n;$ \>$a_1 = 1$ (rekursiv)
\end{tabbing}
\end{itemize}

\subsubsection{Arithmetische Folgen}
Folgen, deren aufeinander folgenden Glieder immer eine konstante Differenz d aufweisen, werden \textcolor{red}{arithmetische Folgen} genannt.\\\\
Darstellungsarten arithmetischer Folgen:
\begin{itemize}
\item $a_n = a_1 + (n-1)d$ (explizit)
\item $a_{n+1} = a_n + d$ (rekursiv)
\end{itemize}
\paragraph{Beispiele:}\hspace{12 cm}
\begin{itemize}
\item $a_n = 2 + (n-1)4$
\item $a_{n+1} = a_n + 3$
\item $1,3,5,7,9,...$
\end{itemize}

\subsubsection{Geometrische Folgen}
Folgen, der Form $a_n = a_1 \cdot q^{n}$, werden als \textcolor{red}{geometrische Folgen} bezeichnet.
Ihre aufeinander folgenden Glieder unterscheiden sich jeweils um einen konstanten Faktor.