Die Integration ist aus der Fl"achenberechnung entstanden. Das Integral ist ein Oberbegriff f"ur das \textbf{bestimmte} und \textbf{unbestimmte} Integral. Die Integration bezeichnet die \textbf{Berechnung} von Integralen. Wir verzichten hier auf den axiomatischen Zugang. 

\subsection{Bestimmtes Integral}
Ein geschlossenes Integral der Funktion $f$ im kompakten (abgeschlossen und beschr"ankt) $\left[ a; b \right]$ ist der Fl"acheninhalt der zwischen der Funktion $f$, der $x$-Achse und der Geraden $x = a$ und $x = b$ eingeschlossen ist. Diese Fl"acheninhalt wird als 
\begin{equation*}
\int\limits_a^b f(x) dx,
\end{equation*}
gelesen: Integral von $a$ bis $b$ von $f \ dx$.

\subsection{Unbestimmtes Integral / Stammfunktion}
In gewissen Sinne ist die Integration die Umkehrung der Differentiation. So ist $F$ die \textbf{Stammfunktion} der Funktion $f$, wenn
\begin{equation*}
F' = f.
\end{equation*}
$F$ ist. $F$ nennen wir auch unbestimmtes Integral. Hin und wieder wird mit dem unbestimmten Integral auch die Menge der Stammfunktionen von $f$ bezeichnet.
\paragraph{Beispiel:} Sei $f(x) = x^2$, so ist $F(x) = \frac{1}{3} x^3$ ein (nicht das) unbestimmtes Integral von $f$. Beweis: $F'(x) = 3 \cdot \frac{1}{3} \cdot x^{3-1} = f(x)$. Allerdings sind alle Funktionen aus der Menge 
\begin{equation*}
\mathcal{F} := \left\{ F \ | \ F(x) = \frac{1}{3} x^3 + c : c \in \mathbb{R} \right\}
\end{equation*}
eine Stammfunktion von $f$.

\subsection{Hauptsatz der Differential- und Integralrechnung}
Dieser Satz stellt eine Beziehung zwischen Stammfunktion und Integral her: Sei $f$ eine auf $\left[a; b\right]$ stetige Funktion und $F$ eine Stammfunktion von $f$, so gilt:
\begin{equation*}
\int\limits_a^b f(x) dx = F(b) - F(a)
\end{equation*}
 

%\subsubsection{Unbestimmtes Integral}
%Definition (Integral): ein Integral ist die Fl"achenbilanz zwischen einem Funktionsgraphen und der x-Achse.\\
%Integrale werden immer in Intervallen berechnet.\\
%Dabei unterscheidet man offene Integrale, d.h ein Integral ohne Grenzen und ein geschlossenes Integral, welches lediglich in einem bestimmten Intervall berechnet wird.\\
%Schreibweise: $\int f(x)dx\Rightarrow$  offenes Integral\\
%\hspace{2.5 cm}$\int\limits_{a}^{b}f(x)dx \Rightarrow$ geschlossenes Integral im Intervall [a;b]\\
%Dabei beschreibt das dx am Ende jeweils, nach welcher Variable integriert wird. In der Physik beispielsweise findet man auch h"aufig die Bezeichnung dt.
%\subsubsection{Stammfunktionen}
%Definition: die Ableitung einer Stammfunktion ergibt den Funktionsterm der urspr�nglichen Funktion.\\
%Symbol der Stammfunktion: F(x)     d.h. $F'(x)=f(x)$

\subsection{Berechnung von Stammfunktionen}
Die Stammfunktion erh"alt man, indem man die Ableitung r"uckw"arts rechnet. D.h. ich addiere den Exponenten um 1, bilde daraus einen Bruch der Form $\frac{1}{Exponent+1}$ und multipliziere diesen mit dem Faktor vor der Basis.
Mit dieser Methode lassen sich zu fast allen Polynomen Stammfunktionen berechnen. Da wir hierbei eine Menge $\mathcal{F}$ an Funktionen berechnen m"ussen wir noch festlegen welche Funktion, wir genau meinen. Wir legen daf"ur das $c$ (siehe obiges Beispiel) fest.

%\subsubsection{Hauptsatz der Differential- und Integralrechnung (HDI)}
%\textcolor{red}{HDI}:\textcolor{red}{!!!Jede Integralfunktion ist eine Stammfunktion der zu integrierenden Funktion!!!}

\subsection{Berechnung von Integralen}
Wir benutzen den Hauptsatz der Differential- und Integralrechnung um ein Integral $\int_a^b f(x) dx$ zu berechnen:
\begin{enumerate}
\item Berechne $F$, sodass $F' = f$
\item Berechne $F(b)-F(a)$
\end{enumerate}

\subsection{Berechnung von Fl"ache zwischen zwei Graphen}
Um die Fl"ache zwischen zwei Graphen $f(x)$ und $g(x)$ zu ermitteln, berechnet man sich zun"achst die Schnittpunkte der beiden Graphen, welche die Intervallgrenzen bilden. Wenn man nun eine neue Funktion $h(x)$ aus der Subtraktion der beiden Funktionen $f(x)$ und $g(x)$ bildet und das Integral dieser Funktion "uber die beiden Schnittpunkte berechnet, erh"alt man die Fl"ache zwischen den beiden Graphen.

\subsection{Das Riemann-Integral (Bonus)}
Um den Hauptsatz der Differential- und Integralrechnung anwenden zu k"onnen muss $f$ in $\left[a;b\right]$ stetig sein. Gilt dies nicht, so k"onnen wir das Integral nicht ohne weiteres Berechnen. Beim Riemann Integral reduziert man die Berechnung des Integrals auf die Berechnung von vielen, einfach zu berechnenden Rechtecken. Dabei wird das Intervall $\left[a;b \right]$ in $n$ Intervalle $I_i = \left[x_{i-1}, x_i \right]$ aufgeteilt. Zudem gibt es f"ur jedes Intervall eine Zwischenstelle $t_i \in I_i$. Dabei gilt $a = x_0, b = x_n$. Die Bedingung ist nun, dass die Zerlegung hinreichend fein gew"ahlt werden kann. Es folgt dann
\begin{equation*}
\int\limits_a^b f(x) dx \approx \sum\limits_{i=1}^n f(t_i) \cdot (x_{i} - x_{i-1})
\end{equation*}
und f"ur $n \to \infty$
\begin{equation*}
\int\limits_a^b f(x) dx = \sum\limits_{i=1}^\infty f(t_i) \cdot (x_{i} - x_{i-1})
\end{equation*}
Es existieren Funktionen die nicht riemannintegrierbar sind. Ein weiteres Integral, welches allerdings nicht in der Vorlesung besprochen wird, ist das sog. Lebesgue-Integral.

\subsection{"Ubung}
\begin{enumerate}
\item Berechne die Stammfunktionen folgender Funktionen
\begin{itemize}
\item $f(x)=x^2-8$
\item $f(x)=3x^2+4x-16$
\item $f(x)=sin(x)-3$
\end{itemize}
\item Berechne nun alle Fl"achen zwischen den oben gegebenen Graphen.
\end{enumerate}